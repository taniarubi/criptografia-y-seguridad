\documentclass[letterpaper,11pt]{article}
% Soporte para los acentos.
\usepackage[utf8]{inputenc}
\usepackage[T1]{fontenc}
% Idioma español.
\usepackage[spanish,mexico, es-tabla]{babel}
% Soporte de símbolos adicionales (matemáticas)
\usepackage{multirow}
\usepackage{amsmath}
\usepackage{amssymb}
\usepackage{amsthm}
\usepackage{amsfonts}
\usepackage{latexsym}
\usepackage{enumerate}
\usepackage{ragged2e}
\usepackage{listings}
\usepackage{xcolor}
\usepackage{multirow} % para las tablas
\usepackage{multicol}
% Modificamos los márgenes del documento.
\usepackage[lmargin=2cm,rmargin=2cm,top=2cm,bottom=2cm]{geometry}

\definecolor{codegreen}{rgb}{0,0.6,0}
\definecolor{codegray}{rgb}{0.5,0.5,0.5}
\definecolor{codepurple}{rgb}{0.58,0,0.82}
\definecolor{backcolour}{rgb}{0.95,0.95,0.92}

\lstdefinestyle{mystyle}{
    backgroundcolor=\color{backcolour},   
    commentstyle=\color{codegreen},
    keywordstyle=\color{magenta},
    numberstyle=\tiny\color{codegray},
    stringstyle=\color{codepurple},
    basicstyle=\ttfamily\footnotesize,
    breakatwhitespace=false,         
    breaklines=true,                 
    captionpos=b,                    
    keepspaces=true,                 
    numbers=left,                    
    numbersep=5pt,                  
    showspaces=false,                
    showstringspaces=false,
    showtabs=false,                  
    tabsize=2
}

\lstset{style=mystyle}

\title{Criptografía y Seguridad \\ Tarea 1}
\author{Altamirano Vázquez Jesús Fernando \\
        Rubí Rojas Tania Michelle}
\date{14 de marzo de 2020}

\begin{document}
\maketitle

\begin{enumerate}
    % Ejercicio 1.
    \item Mostrar que si $(a, n) = 1$, los conjuntos de clases 
    $\{a,2a,3a,...,(n-1)a\} = \{1,2,...,n-1\}$ en $\mathbb{Z}_{n}$.
    \begin{proof}
        Supongamos que $(a, n) = 1$. Por hipótesis, sabemos que cada entero es
        congruente $\pmod{n}$ a exactamente uno de $a, 2a, 3a, ..., (n-1)a$.
        Ahora bien, el conjunto $\{a,2a,3a,...,(n-1)a\}$ tiene $n-1$ elementos,
        y ninguno de ellos es congruente con $0 \pmod{n}$, por lo que cada uno
        de los elementos es congruente $\pmod{n}$ a uno de los números del
        conjunto $\{a,2a,3a,...,(n-1)a\}$. 
        
        Debemos mostrar que no hay dos enteros en el conjunto 
        $\{a,2a,3a,...,(n-1)a\}$ que sean congruentes $\pmod{n}$, ya que se
        sigue que sus residuos \textit{mínimos} $\pmod{n}$ son todos diferentes 
        y eso hace que sea igual al conjunto $\{1,2,...,n-1\}$. 
        
        Supongamos que hay dos enteros en el conjunto $\{a,2a,3a,...,(n-1)a\}$ 
        que son congruentes $\pmod{n}$, esto es, $ka \equiv la \pmod{n}$. Como
        $(a, n) = 1$, por el teorema 
        \begin{center}
            Si $ac \equiv bc$ y $(c,n) = d$, entonces $a \equiv b 
            \pmod{\frac{m}{d}}$
        \end{center}
        
        tenemos que $k \equiv l \pmod{n}$, lo que implica que $k = l$.
        
        Por lo tanto, $\{a,2a,3a,...,(n-1)a\} = \{1,2,...,n-1\}$.
    
    \end{proof}

    
    % Ejercicio 2.
    \item Dar las unidades de $\mathbb{Z}_{156}$ y su inverso multiplicativo de
    la siguiente forma $(a, a^{-1})$.
    
    \textsc{Solución:} Sabemos que un elemento $a$ es invertible módulo $m$ si
    existe $a^{-1}$ en $\mathbb{Z}_{m}$ tal que $a \cdot a^{-1} = 1$. Decimos
    que $a^{-1}$ es el inverso de $a$ en $Z_{m}$. Una unidad es un elemento
    $x$ que es invertible módulo $m$. Para calcular todas las unidades de 
    $\mathbb{Z}_{156}$ lo que hicimos fue implementar el algoritmo extendido de
    Euclides en Python junto con una función $unidades()$ que se encarga de 
    calcular las unidades.
    
    Para la implementación del $AEE$ simplemente seguimos el procedimiento que
    se nos enseñó en clase. Ahora bien, para calcular las unidades tuvimos
    en cuenta el siguiente teorema
    \begin{center}
        \textbf{Un entero $a$ es invertible módulo $m$ si y sólo si 
        $(a,m) = 1$. Si $a$ posee inverso, entonces éste es único.}
    \end{center}
   
    Para calcular el inverso, entonces lo que hacemos es
    \begin{itemize}
        \item Aplicamos el $AEE$ para calcular $(a, m)$. Si es diferente de $1$,
        entonces no es invertible. Si es igual a $1$, entonces obtenemos la
        identidad de \textit{Benzout} : $as + mt = 1$.
        
        \item En $Z_{m}$, $a \cdot s = 1$, lo que implica que $s$ es el inverso
        de $a$ módulo $m$.
    \end{itemize}
    
    Aplicamos este procedimiento para cada uno de los elementos de 
    $\mathbb{Z}_{156}$, y obtenemos todas las unidades. 

    \begin{lstlisting}[language=Python]
        '''
        Regresa una tupla (mcd, s, t) que obtenemos al aplicar el algoritmo 
        extendido de Euclides, donde as + bt = mcd(a, b) son los elementos que 
        conforman la tupla.
        '''
        def aee(a, b):
            s = 0; s_i = 1
            t = 1; t_i = 0
            g = b; g_i = a

            while g != 0:
                cociente = g_i // g
                g_i, g = g, g_i - cociente * g
                s_i, s = s, s_i - cociente * s
                t_i, t = t, t_i - cociente * t
                
            return (g_i, s_i, t_i)

        # Calcula todas las unidades en Z_{156}.
        def unidades():
            for i in range(0, 156):
                g, s, t = aee(i, 156)
                # El inverso multiplicativo de a modulo m existe sii (a,m) = 1
                if g != 1:
                    continue 
                else:
                    inverso = s % 156
                    print("(" + str(i) + ", " + str(inverso) + ")")
            
        if __name__ == "__main__":
            unidades()
    \end{lstlisting}
    
    Después de ejecutar el programa, obtenemos $48$ unidades, las cuales son:
    \begin{table}[h]
        \begin{center}
            \begin{tabular}{llllll}
                (1, 1) & (5, 125) & (7, 67) & (11, 71) & (17, 101) & 
                (19, 115)\\ (23, 95) & (25, 25) & (29, 113) & (31, 151) &
                (35, 107) & (37, 97) \\ (41, 137) & (43, 127) & (47, 83) &
                (49, 121) & (53, 53) & (55, 139) \\ (59, 119) & (61, 133) & 
                (67, 7) & (71, 11) & (73, 109) & (77, 77) \\ (79, 79) & 
                (83, 47) & (85, 145) & (89, 149) & (95, 23) & (97, 37) \\
                (101, 17) & (103, 103) & (107, 35) & (109, 73) & (113, 29) &
                (115, 19) \\ (119, 59) & (121, 49) & (125, 5) & (127, 43) &
                (131, 131) & (133, 61) \\ (137, 41) & (139, 55) & (145, 85) &
                (149, 89) & (151, 31) & (155, 155) 
            \end{tabular}
            \caption{Unidades en $Z_{156}$}
        \end{center}
    \end{table}
   
    % Ejercicio 3.
    \item De los siguientes sistemas de congruencias decir si tienen solución,
    y en caso de tenerla, dar la solución.
    \begin{itemize}
        % Ejercicio 3.a
        \item[a)]
        \begin{align*}
            x &\equiv 10 \pmod{65} \\
            x &\equiv 25 \pmod{85} \\
            x &\equiv 35 \pmod{70} \\
            x &\equiv 15 \pmod{35}
        \end{align*}
        
        \textsc{Solución:} Por el \textit{Teorema Chino del Residuo} sabemos que
        un sistema de congruencias lineales tiene solución si y sólo si para
        cualesquiera $i, j = 1,...,k, (m_{i}, m_{j}) \; | \; a_{i} - a_{j}$;
        donde $m_{i}, m_{j}$ son los módulos y $a_{i}, a_{j}$ son los enteros
        en el lado derecho de la congruencia.
        
        Tenemos que $(65,85) = 5 \; | -15 = 10-25$, $(65, 70) = 5 \; | -25 =
        10-35$, \\ $(65, 35) = 5 \; | -5 = 10-15$, $(85, 70) = 5 \; | -10 = 
        25-35$, $(85, 35)= 5 \; | \; 10 =25-15$ pero $(70, 35) = 35  \not | \;
        20 = 35-15$. Por este último resultado, podemos concluir que el sistema
        no tiene solución.
        
        % Ejercicio 3.b
        \item[b)]
        \begin{align}
            x &\equiv 15 \pmod{35} \\
            x &\equiv 10 \pmod{65} \\
            x &\equiv 25 \pmod{85} \\
            x &\equiv 15 \pmod{145}
        \end{align}
        
        \textsc{Solución:} Por el \textit{Teorema Chino del Residuo} sabemos que
        un sistema de congruencias lineales tiene solución si y sólo si para
        cualesquiera $i, j = 1,...,k, (m_{i}, m_{j}) \; | \; a_{i} - a_{j}$;
        donde $m_{i}, m_{j}$ son los módulos y $a_{i}, a_{j}$ son los enteros
        en el lado derecho de la congruencia.
        
        Tenemos que $(35,65) = 5 \; | \; 5 = 15-10$, $(35,85) = 5 \; | -10 
        = 15-25$, \\ $(35,145) = 5 \; | \; 0 = 15-15$, $(65,85) = 5 \; | -15
        = 10-25$, $(65,145) = 5 \; | -5 = 10-15$ y $(85, 145) = 5 \; | \; 10
        = 25-15 $. Por lo tanto, el sistema de congruencias tiene solución.
        
        Ahora, resolveremos primero las congruencias $(1)$ y $(2)$. Las
        soluciones de $x \equiv 15 \pmod{35}$ están dadas por 
        \begin{align}
            x = 15 + 35y, \; y \in \mathbb{Z}
        \end{align}
        
        Veamos para cuáles valores de $y$, $x$ también es solución de la
        segunda congruencia: sustituimos $x$ en la segunda ecuación y nos
        queda 
        \begin{align*}
            15 + 35y \equiv 10 \pmod{65} 
        \end{align*}
        
        es decir, 
        \begin{align*}
            35y \equiv 10-15 = -5 \pmod{65}
        \end{align*}
        
        la cual tiene las mismas soluciones que la congruencia
        \begin{align}
            7y \equiv -1 \pmod{13}
        \end{align}
        
        Dado que $2$ es inverso multiplicativo de $7 \pmod{13}$ (ya que 
        $7 \cdot 2 \equiv 1 \pmod{13}$), multiplicando por $2$ la congruencia
        $(6)$ tenemos que
        \begin{align*}
            y \equiv 11 \pmod{13} 
        \end{align*}
        
        por lo que
        \begin{align}
            y = 11 + 13z, \; z \in \mathbb{Z}
        \end{align}
        
        Sustituyendo $(7)$ en $(5)$ obtenemos que conjunto de soluciones 
        simultáneas de las congruencias $(1)$ y $(2)$ es 
        \begin{align*}
            x 
            &= 15 + 35(11 + 13z) \\
            &= 400 + 455z\; z \in \mathbb{Z}
        \end{align*}
        
        o, lo que es equivalente
        \begin{align*}
            x \equiv 400 \pmod{455}
        \end{align*}
        
        Por lo que, el nuevo sistema a resolver es
        \begin{align}
            x &\equiv 400 \pmod{455} \\
            x &\equiv 25 \pmod{85} \\
            x &\equiv 15 \pmod{145}
        \end{align}
        
        Postetiormente, resolveremos las congruencias $(9)$ y $(10)$. Las 
        soluciones de $x \equiv 25 \pmod{85}$ están dadas por
        \begin{align}
            x = 25 + 85p, \; p \in \mathbb{Z}
        \end{align}
        
        Veamos ahora para cuáles valores de $p$, $x$ también es solución de la 
        segunda congruencia: sustituimos $x$ en la segunda ecuación y nos queda
        \begin{align*}
            25 + 85p \equiv 15 \pmod{145}
        \end{align*}
        
        es decir,
        \begin{align*}
            85p \equiv 15 - 25 = -10 \pmod{145}
        \end{align*}
        
        la cual tiene las mismas soluciones que la congruencia 
        \begin{align}
            17p \equiv -2 \pmod{29}
        \end{align}
        
        Dado que $12$ es inverso multiplicativo de $17 \pmod{29}$ (ya que 
        $12 \cdot 17 \equiv 1 \pmod{29}$), multiplicando por $12$ la congruencia
        $(12)$, tenemos que 
        \begin{align*}
            p \equiv 5 \pmod{29} 
        \end{align*}
        
        por lo que 
        \begin{align}
            p = 5 + 29q, \; q \in \mathbb{Z}
        \end{align}
        
        Sustituyendo $(13)$ en $(11)$ obtenemos que el conjunto de soluciones
        simultáneas de las congruencias $(9)$ y $(10)$ es
        \begin{align*}
            x 
            &= 25 + 85(5 + 29q) \\
            &= -2015 + 2465q, \; q \in \mathbb{Z}
        \end{align*}
        
        o, lo que es equivalente
        \begin{align*}
            x \equiv 450 \pmod{2465}
        \end{align*}
        
        Por lo que, el nuevo sistema a resolver es
        \begin{align}
            x &\equiv 400 \pmod{455} \\
            x &\equiv 450 \pmod{2465}
        \end{align}
        
        Finalmente, resolvemos las congruencias $(14)$ y $(15)$. Las soluciones 
        de $x \equiv 400 \pmod{455}$ están dadas por
        \begin{align}
            x = 400 + 455r, \; r \in \mathbb{Z}
        \end{align}
        
        Veamos ahora para cuáles valores de $r$, $x$ también es solución de la
        segunda congruencia: sustituimos $x$ en la segunda ecuación y nos
        queda
        \begin{align*}
            400 + 455r \equiv 450 \pmod{2465}
        \end{align*}
        
        es decir,
        \begin{align*}
            455r \equiv 450-400 = 50 \pmod{2465}
        \end{align*}
        
        la cual tiene las mismas soluciones que la congruencia 
        \begin{align}
            91r \equiv 10 \pmod{493}
        \end{align}
        
        Dado que $428$ es inverso multiplicativo de $91 \pmod{493}$ (ya que
        $428 \cdot 91 \equiv 1 \pmod{493}$), multiplicando por $428$ la 
        congruencia $(17)$, tenemos que 
        \begin{align*}
            r \equiv 336 \pmod{493}
        \end{align*}
        
        por lo que 
        \begin{align}
            r = 336 + 493t, \; t \in \mathbb{Z}
        \end{align}
        
        Sustituyendo $(18)$ en $(16)$ obtenemos que el conjunto de soluciones
        simultáneas de las congruencias $(14)$ y $(15)$ es
        \begin{align*}
            x 
            &= 400 + 455(336 + 493t) \\
            &= 153280 + 224315t, \; t \in \mathbb{Z}
        \end{align*}
        
        Por lo tanto, la solución del sistema de congruencias original 
        $(1), (2), (3), (4)$ es:
        \begin{align*}
            x \equiv 153280 \pmod{224315}
        \end{align*}
    \end{itemize}
    
    % Ejercicio 4.
    \item Dado el siguiente texto cifrado
    \begin{verbatim}
    ORNOQM PTO ORSO KOLRJFO IOR JNMQSO KJR OL IM PTO GDEO, PTO
    OI GOREDAQJQIM. ORSJL OLSQJLGM JI KTLGM GO IJ EQDNSMBQJADJ Y IJ
    ROBTQDGJG Y NJQJ JPTOIIMR PTO ORSOL DLSOQORJGMR OL IJ EQDNSMBQ-
    JADJ IOR NQMNMLBM PTO AMQKOL TL BQTNM Y ORSO IM GDVDGJL OL
    GMR RTUBQTNMR TLM EDAQJQJ Y OI MSQM RTUBQTNM GOREDAQJQJ. OI
    QOSM OR OI RDBTDOLSO: OI RTUBQTNM PTO EDAQJ OLEQDNSJ TL KOLRJ-
    FO Y IM OLVDJ EDAQJGM J OI RTUBQTNM PTO GOREDAQJ GDEDOLGM PTO
    EDAQJGM TRM Y LM KJR DLAMQKJEDML. RD OI BQTNM PTO GOREDAQJ
    SQJGTEO OI KOLRJFO RO FTLSJL Y RJEJL EMLEITRDMLOR GO PTO AJISJ
    NJQJ KOFMQJQ OI EDAQJGM, OL EJRM EMLSQJQDM OI RTUBQTNM PTO
    EDAQJ JNMYJ JI PTO GOREDAQJ NJQJ PTO JVJLEO OI BQTNM. EMKDOLEOL
    EML IMR EDAQJGMR KJR ROLEDIIMR EMKM KMLMJIAJUOSDEMR, GORN-
    TOR IMR NMIDJIAJUOSDEMR, ITOBM IMR EDAQJGMR OL UIMPTOR EMKM
    CDII Y JRD. OR KTY DKNMQSJLSO SOLOQ OL ETOLSJ PTO RML TL BQTNM Y
    PTO GOUOL JNMYJQRO OLSQO RD. ORSJR NQJESDEJR IOR GJQJL KJGTQOZ
    Y OXNOQDOLEDJ, OI RDBTDOLSO NJRM OR DKNIOKOLSJQIM OL IJ VDGJ
    EMSDGDJLJ, NMQ OFOKNIM GJGM ETJIPTDOQ JQECDVM, AQJBKOLSJQIM,
    EDAQJQIM Y GOFJLGMIM OL OI KDRKM AMQKJSM, GORNTOR IJ NJQSO
    PTO GOREDAQJ, DKNIOKOLSJ OI JIBMQDSKM GO GOREDAQJGM GOFJLGM
    OI JQECDVM EMKM OI MQDBDLJI. QOETOQGOL PTO OL ORSJ NJQSO OR
    KTY DKNMQSJLSO PTO TRSOGOR CJBJL SMGMR IMR NQMBQJKJR SJLSM
    NJQJ EDAQJQ EMKM NJQJ GOREDAQJQ, YJ PTO OI METNJQ RMAWJQO GO
    SOQEOQMR EMKNQMKOSO SMGM OI SQJUJFM. TLJ VOZ COECM ORSM RO
    GJQJL ETOLSJ GO PTO KTECJR NOQRMLJR LOEORDSJL GO RTR ROQV-
    DEDMR Y JI CJEOQ ORSJR NQJESDEJR OL OI AMLGM RO ORSJL NQONJQJLGM
    KOQEJGM IJUMQJI Y LM EMKM OKNIOJGMR RDLM EMKM OKNQORJQDMR.
    IJ VOLSJFJ GO CJEOQIM GO ORSJ KJLOQJ, OR PTO OL OI KOQEJGM JESTJI
    IJR NOPTOLJR OKNQORJR LOEORDSJL GO TRSOGOR NJQJ EQOEOQ LOEORDSJLGM
    GO TLJ EQDNSMBQJADJ KJR NOQRMLJIDZJGJ Y KOLMR EMKOQEDJI UQDLGJLGM
    JRD KJR EMLADJLZJ OL IJR OKNQORJR M NOQRMLJR PTO EMLSQJSJL LTORSQMR
    ROQVDEDMR, NMQPTO IJR BQJLGOR OKNQORJR PTO JESTJIKOLSO UQDLGJL
    ORO ROQVDEDM CJL AJIIJGM. NMQ OFOKNIM OL OI GMR KDI PTDLEO OI
    OREJLGJIM GO ORNDMLJFO NMQ NJQSO GO BMMBIO J NOQRMLJR Y OKN-
    QORJR GO IJ EMKTLDGJG OTQMNOJ, IJR ETJIOR QONOQETSDOQML SJLSM
    NMIDSDEJKOLSO EMKM OEMLMKDEJKOLSO, ORSO SDNM GO JEMLSOEDK-
    DOLSMR JUQO NTOQSJR NJQJ NOQRMLJR EMKM LMRMSQMR YJ PTO IJ GO-
    REMLADJLZJ GO IJR BQJLGOR OKNQORJR PTO RO GOGDEJL J IJ ROBTQDGJG
    EMKNTSJEDMLJI RO CJ NTORSM OL SOIJ GO FTDEDM, NMQ ORM OR DKNMQSJL-
    SO PTO GORGO JCMQDSJ EMKDOLEOL J SQJUJFJQ IJR NOQRMLJR PTO
    ORSJL DLSOQORJGJR. LM ROQJ AJEDI, NOQM LJGJ PTO VJIBJ IJ NOLJ OR
    AJEDI. RJITGMR Y UDOLVOLDGMR JI ETQRM ORNOQJKMR PTO IM GDRAQT-
    SOL.
    \end{verbatim}
    
    \begin{enumerate}
        % Ejercicio 4.a
        \item Hacer análisis de frecuencias.
        
        \textsc{Solución:}
        Realizamos el análisis de frecuencias de cada una de las letras 
        (manualmente) y colocamos los resultados obtenidos en una bonita tabla:

        \begin{center}
        \begin{tabular}{|c|c|c|}
        \hline
        Letra & No. Apariciones & Frecuencia \\
        \hline 
        A & 36 & 1.7$\%$ \\ \hline
        B & 28 & 1.3$\%$ \\ \hline
        C & 12 & 0.5$\%$ \\ \hline
        D & 111 & 5.24$\%$ \\ \hline
        E & 101 & 4.7$\%$ \\ \hline 
        F & 14 & 0.6$\%$ \\ \hline 
        G & 90 & 4.3$\%$ \\ \hline 
        H & 0 & 0.0$\%$ \\ \hline 
        I & 99 & 4.7$\%$ \\ \hline 
        J & 242 &11.6$\%$ \\ \hline 
        K & 65 & 3.1$\%$ \\ \hline 
        L & 134 & 6.4$\%$ \\ \hline 
        M & 184 & 8.8$\%$ \\ \hline
        N & 78 & 3.7$\%$ \\ \hline 
        O & 294 & 14.1$\%$ \\ \hline
        \end{tabular}
        \end{center}

        \begin{center}
        \begin{tabular}{|c|c|c|}
        \hline
        Letra & No. Apariciones & Frecuencia \\
        \hline
        P & 35 &1.6$\%$ \\ \hline 
        Q & 151 & 7.2$\%$ \\ \hline 
        R & 169 & 8.1$\%$ \\ \hline 
        S & 87 & 4.1$\%$ \\ \hline 
        T & 93 & 4.4$\%$ \\ \hline 
        U & 16 & 0.7$\%$ \\ \hline 
        V & 13 & 0.6$\%$ \\ \hline 
        W & 1 & 0.05$\%$ \\ \hline 
        X & 1 & 0.05$\%$ \\ \hline 
        Y & 22 & 1.0$\%$ \\ \hline 
        Z & 5 & 0.2$\%$ \\ \hline 
        \end{tabular}
        \end{center}
        
        % Ejercicio 4.b
        \item Dar la clave de cifrado.
        \begin{verbatim}
        A B C D E F G H I J K L M N O P Q R S T U V W X Y Z
        J U E G O A B C D F H I K L M N P R S T U V W X Y Z
        \end{verbatim}
        
        \textsc{Solución:} Juego

        % Ejercicio 4.c
        \item Dar la regla de descifrado.
        
        \textsc{Solución:} Tomemos en cuenta las letras más usadas en español

        \begin{center}
        \begin{tabular}{|c|c|}
        \hline
        Letra & Frecuencia \\
        \hline 
        E & 13.68$\%$ \\ \hline 
        A & 12.53$\%$ \\ \hline 
        O & 8.68$\%$ \\ \hline 
        S & 7.98$\%$ \\ \hline 
        R & 6.87$\%$ \\ \hline 
        N & 6.71$\%$ \\ \hline 
        I & 6.25$\%$ \\ \hline 
        \end{tabular}
        \end{center}
        
        Realizamos una sustitución de las letras mas usadas del español con las
        más usadas en el texto, haremos esto en los primeros tres renglones del 
        texto cifrado, de donde obtenemos:
        
        \begin{verbatim}
        ESNERO PTE ESSE KENSAFE IES ANORSE KAS EN IO PTE GIEE, PTE EI
        GESEIARARIO. ESSAN ENSRANGO AI KTNGO GE IA ERINSOBRAAIA Y IA
        SEBTRIGAG Y NARA APTEIIOS PTE ESSEN INSERESAGOS EN IA ERINSOBRAAIA
        \end{verbatim}
        
        Entonces
        \begin{center}
            O $\rightarrow$ E \\
            J $\rightarrow$ A \\
            M $\rightarrow$ O \\
            R $\rightarrow$ S \\
            Q $\rightarrow$ R \\
            L $\rightarrow$ N \\
            D $\rightarrow$ I
        \end{center}
        
        \newpage
        De este fragmento podemos deducir que 
        \begin{center}
            N $\rightarrow$ P \\
            S $\rightarrow$ T \\
            K $\rightarrow$ M \\
            F $\rightarrow$ J \\
            I $\rightarrow$ L \\
            G $\rightarrow$ D \\
            Y $\rightarrow$ Y
        \end{center}
        
        Con esto obtenemos que 
        \begin{verbatim}
        ESPERO PTE ESTE MENSAJE LES APORTE MAS EN LO PTE DIEE, PTE EL
        DESEIARARLO. ESTAN ENTRANDO AL MTNDO DE LA ERIPTOBRAAIA Y LA
        SEBTRIDAD Y PARA APTELLOS PTE ESTEN INTERESADOS EN LA ERIPTOBRAAIA
        \end{verbatim}
        
        Por contexto, deducimos que 
        \begin{center}
            P $\rightarrow$ Q \\
            T $\rightarrow$ U \\
            E $\rightarrow$ C \\
            B $\rightarrow$ G \\
            A $\rightarrow$ F
        \end{center}
        
        de donde obtenemos
        \begin{verbatim}
        ESPERO QUE ESTE MENSAJE LES APORTE MAS DE LO QUE DICE, QUE EL
        DESCIFRARLO. ESTAN ENTRANDO AL MUNDO DE LA CRIPTOGRAFIA Y LA
        SEGURIDAD Y PARA AQUELLOS QUE ESTEN INTERESADOS EN LA CRIPTOGRAFIA
        \end{verbatim}
        
        Sustituyendo en el texto tenemos que 
        \begin{center}
            C $\rightarrow$ H \\
            U $\rightarrow$ B \\
            V $\rightarrow$ B \\
            W $\rightarrow$ W \\
            X $\rightarrow$ X \\
            Z $\rightarrow$ Z
        \end{center}

        % Ejercicio 4.d
        \item Descifrar el mensaje.
        
        \textsc{Solución:}
        \begin{verbatim}
        ESPERO QUE ESTE MENSAJE LES APORTE MAS EN LO QUE DICE QUE EL 
        DESCIFRARLO ESTAN ENTRANDO AL MUNDO DE LA CRIPTOGRAFIA Y LA 
        SEGURIDAD Y PARA AQUELLOS QUE ESTEN INTERESADOS EN LA CRIPTOGRAFIA 
        LES PROPONGO QUE FORMEN UN GRUPO Y ESTE LO DIVIDAN EN DOS SUBGRUPOS 
        UNO CIFRARA Y EL OTRO SUBGRUPO DESCIFRARA EL RETO ES EL SIGUIENTE 
        EL SUBGRUPO QUE CIFRA ENCRIPTA UN MENSAJE Y LO ENVIA CIFRADO A EL 
        SUBGRUPO QUE DESCIFRA DICIENDO QUE CIFRADO USO Y NO MAS INFORMACION     
        SI EL GRUPO QUE DESCIFRA TRADUCE EL MENSAJE SE JUNTAN Y SACAN 
        CONCLUSIONES DE QUE FALTA PARA MEJORAR EL CIFRADO EN CASO CONTRARIO 
        EL SUBGRUPO QUE CIFRA APOYA AL QUE DESCIFRA PARA QUE AVANCE EL     
        GRUPO COMIENCEN CON LOS CIFRADOS MAS SENCILLOS COMO MONOALFABETICOS 
        DESPUES LOS POLIALFABETICOS LUEGO LOS CIFRADOS EN BLOQUES COMO HILL 
        Y ASI ES MUY IMPORTANTE TENER EN CUENTA QUE SON UN GRUPO Y QUE 
        DEBEN APOYARSE ENTRE SI ESTAS PRACTICAS LES DARAN MADUREZ Y 
        EXPERIENCIA EL SIGUIENTE PASO ES IMPLEMENTARLO EN LA VIDA COTIDIANA 
        POR EJEMPLO DADO CUALQUIER ARCHIVO FRAGMENTARLO CIFRARLO Y DEJANDOLO 
        EN EL MISMO FORMATO DESPUES LA PARTE QUE DESCIFRA IMPLEMENTA EL 
        ALGORITMO DE DESCIFRADO DEJANDO EL ARCHIVO COMO EL ORIGINAL 
        RECUERDEN QUE EN ESTA PARTE ES MUY IMPORTANTE QUE USTEDES HAGAN 
        TODOS LOS PROGRAMAS TANTO PARA CIFRAR COMO PARA DESCIFRAR YA QUE EL 
        OCUPAR SOFWARE DE TERCEROS COMPROMETE TODO EL TRABAJO UNA VEZ HECHO 
        ESTO SE DARAN CUENTA DE QUE MUCHAS PERSONAS NECESITAN DE SUS 
        SERVICIOS Y AL HACER ESTAS PRACTICAS EN EL FONDO SE ESTAN  
        PREPARANDO MERCADO LABORAL Y NO COMO EMPLEADOS SINO COMO 
        EMPRESARIOS LA VENTAJA DE HACERLO DE ESTA MANERA ES QUE EN EL 
        MERCADO ACTUAL LAS PEQUENAS EMPRESAS NECESITAN DE USTEDES 
        PARA CRECER NECESITANDO DE UNA CRIPTOGRAFIA MAS PERSONALIZADA Y 
        MENOS COMERCIAL BRINDANDO ASI MAS CONFIANZA EN LAS EMPRESAS O 
        PERSONAS QUE CONTRATAN NUESTROS SERVICIOS PORQUE LAS GRANDES 
        EMPRESAS QUE ACTUALMENTE BRINDAN ESE SERVICIO HAN FALLADO POR 
        EJEMPLO EN EL DOS MIL QUINCE EL ESCANDALO DE ESPIONAJE POR PARTE DE 
        GOOGLE A PERSONAS Y EMPRESAS DE LA COMUNIDAD EUROPEA LAS CUALES 
        REPERCUTIERON TANTO POLITICAMENTE COMO ECONOMICAMENTE ESTE TIPO DE 
        ACONTECIMIENTOS ABRE PUERTAS PARA PERSONAS COMO NOSOTROS YA QUE LA 
        DESCONFIANZA DE LAS GRANDES EMPRESAS QUE SE DEDICAN A LA SEGURIDAD 
        COMPUTACIONAL SE HA PUESTO EN TELA DE JUICIO POR ESO ES IMPORTANTE 
        QUE DESDE AHORITA COMIENCEN A TRABAJAR LAS PERSONAS QUE ESTAN 
        INTERESADAS NO SERA FACIL PERO NADA QUE VALGA LA PENA ES FACIL 
        SALUDOS Y BIENVENIDOS AL CURSO ESPERAMOS QUE LO DISFRUTEN.
        \end{verbatim}
    \end{enumerate}
    
    % Ejercicio 5.
    \item Dado el siguiente mensaje cifrado con Vinegere.
    \begin{multicols}{4}
    \begin{verbatim}
    T S I I C H G D E A
    M W J M Z I P O K R
    Q V E J M X R Z C Q
    M O M C L Y A I P L
    W Z X R F E L S S Y
    G S A M H X J C G A
    P C P F J M K J M P
    A W X U Q R G E S P
    O W X Q X A V E E N
    W U O K N Y P K C N
    S K T Q S I G I Z I
    V U G B Y W E F G R
    U C S Z I K M U T W
    B Z N O K M Q L Z Q
    Q P L A C M G D S P
    O J Q W B U K L A A
    C K U F R V X O O E
    G X V G K Z U W Q W
    W D S A M N A Z M G
    U E F A V V I G S L
    X A D E K F M X A J
    Y S Z W P T W E X L
    N W E X B N T A Y Y
    I F A T M P X L G S
    V X Y W E U U X X Z
    W X E P W N E V M H
    J E S C Y F C P F J
    L U G Z L U D L O G
    Z S P M O A S B E O
    E K Q V S I O A K O
    Y W I G O X X D K O
    W I W U M W J M Z I
    V Q R C I N T S D Q
    H A N A V F B Q S M
    G Q C Q M Q I K A W
    P I X K W E J P S X
    R W X G L V E E H F
    Q G N L Q I P K T I
    M X L G G S X S F A
    W X E P Q F E S E G
    N T A Y S A M N Q M
    B V U U K O P X A G
    E C P C N I T D S Q
    X L Z G S X M Z L Z
    M U E G B P X B K C
    E S W T I L A W Z W
    M Q L B T G I V A I
    G J S U Q B A C B A
    Q Q G R J Q D B V N
    D S P Q P E K B I O
    T G P I I X Q E E M
    K C L D M R B T C M
    P L Z Q S G Q P J I
    D A W X U Q R G E S
    O B A M R I W U Q M
    Z I J A F P I B V E
    R V X O O E F F S N
    S G E F R A E A F Q
    W U S G Z P L A S U
    E Y Z G U Y A Q B H
    F F S J C A P W C Y
    T I S K P B G G N V
    F M P D S E I X K G
    M X B Z O I F M H L
    A Y A S Q Z Q P S D
    C U A W X Q F A O O
    I K B Q P G P I J W
    A S Y X T F A V X E
    G E K Q P I Q D R G
    V A K N V G H X M N
    M K J M P T G H M B
    N W D E I W D R W D
    M E E U A Q L C P S
    E O I G L J A W Q Z
    H M D E I E E M Z E
    P O K Y Y V I E E J
    F A G S U D M Q W T
    Z S A M G L D A W B
    S D U F O W U H S Z
    T N E F S Y X R G N
    S A M N C M M P P M
    L A V C C G B S E M
    A L G C Y B D K E F
    I I T A K K I K U K
    Q I P K T I T U V B
    S V Q W M M T T G Y
    M Q R A M H B T C M
    S P K W R K A W N C
    Z M V I I N E K B V
    K T G E Q B F K C S
    S D W L J T E E U W
    O E F L E O M R O J
    I X T I E Q V L L G
    Z I P O W T A Q V O
    O H Q V P W P A D Z
    I F O K B E O I O U
    X F B W L S P S I W
    G D I P M N S A X I
    U I F E S M W T T S
    P A Q N D A W N C G
    Q S I D K D S Z W R
    O F F V X Z P O W Z
    C G H W Y S P T G I
    M K U G D A M X X U
    W M F X V F O F M R
    S U W Z E Z Q Q E F
    B V Q D W E Y I Q D
    M H B C P H W D S B
    R U A E I V W E N A
    M N H A Q V O W R A
    E J I T C S D E I W
    U M R B S T O E B M
    U V W D U R M G M A
    R T V I E M S Y Z C
    F M Q C N A V A I I
    E K O V F B Q R W E
    R Q O S P S I M E E
    S L T K C W Z G F I
    W D U R M U E W Z G
    Q Y E M W Z I T A D
    A M U C J U T Z Q Q
    F E A I R O J A X O
    E K Y I U Q E A F A
    A J U Y A E I J G R
    S Q I F O I G I I I
    G E T B Z O I L Q E
    G S L M I P B W D A
    Z W R A J M Q F K W
    E Y P T K B J A W B
    L W Z K R I L E W X
    M U G M U P I M T M
    M F V V E J Q W M W
    W P M A I U E Z M G
    G I E B E O B K D G
    B O W N L A G R I N
    Z S P
    B A E R A F S O M U
    M P D C P O K P I I
    L I G I B A E R A N
    M B Q M Q R X K K O
    S Q W S U W E Y M W
    W E X O I N E S Y S
    U A E A V L A Q S D
    K K K O E M W C Q P
    N P B T Q S S Y S O
    A T M R A W P A F E
    K Q R S L S K T G S
    G R M P T J M R Y C
    V A T L L G M G E Z
    G N L Q P B A K G M
    L B T O W E G O Q V
    Y I O Q F A Z M E N
    R G F M Q C N A V A
    U C Y S Q P E M T R
    E J W A L W P M G W
    J M Q X Z E A J M P
    U I F P M L A G L S
    L M N H A Q V O W G
    X M L V Q L W M K O
    D W X M Q M T A L G
    T C B W D M K B Q D
    S Z B C V A A T X H
    F F I Z Z K O D X S
    F O W X Q B F K C S
    F M T R S K W B X T
    M W L V T R A E E X
    N V Q R L A Q T J A
    W N A V A R L X W E
    W A M C L Y Q F O I
    L A V F V E O F Y I
    U I E N S I W I I S
    P W E I S D Q B R Q
    M R A W G L K U K R
    A F P S S Q V E G D
    A W N E M R B R Q D
    I P X C N G X I I C
    G S V X J K N K W C
    T E D M R X T K S A
    L V X E J F M A W G
    M R Q M U D W F I O
    E K E S K A W S S G
    \end{verbatim}
    \end{multicols}
    
    \begin{enumerate}
        % Ejercicio 5.a
        \item Hacer la prueba de KasisKi describiéndo los pasos para el mensaje.
        
        \textsc{Solución}:
        Lo primero que debemos hacer para decifrar nuestro texto es averiguar la
        longitud de la cadena. Al momento de hacer análisis sobre el texto nos 
        damos cuenta de algunas palabras con su número de repeticiones y su 
        longitud. En particular, tenemos que 
        \begin{itemize}
            \item Dos cadenas $MKJMPTG$, una de ellas se encuentra en la posición 
            $245$ y la otra en  la posición $1050$.

            \item Dos cadenas $PII$ en las posiciones $77$ y $252$.

            \item Dos cadenas $DURM$ en las posiciones $903$ y $1141$.
        \end{itemize}
        
        Nos percatamos de que el máximo común divisor de $77, 245, 252, 904, 
        1050$ y $1141$ es $7$, entonces la longitud de nuestra cadena 
        probablemente sea de $7$.
        
        Ahora haremos $7$ subcriptogramas, en donde tomaremos el caracter $0$ 
        con el $7, 14, ...$ etc y el $1$ con el $8$, con el $15$, etc. Así, 
        hasta llegar a $6$. Entonces 

        \begin{verbatim}
        [1.] TDBDUFJRALMPMQAQGNAUZQQZSQQZEMDBBE
        YMPMPAAEMDXMXNQENYNAEAMEGMQZLUGPFZ
        MZRFYAEMEQMQQDZMMEQAAZYMRFCEFAUK
        MDQDMEAAPAMMQMMFADDPYEXMBQEYPD
        XMMSXMXGUQFADXMGOEACMHPFXUDDZXZBAD
        ZKQOAMMEOMZUQAJYFAAQDDYAQAZAAAYQUZG
        NPPSEQDXFMMMUQAEMPDXCEAMXEKZXXYQU
        USWOQYQMEDMPMFUXABMFDMBAEGMEZ    
        \end{verbatim}

        \begin{verbatim}
        [2.] SEPWWSMEWEPIXLIVIMIQIVRXYDVSYHSIEXS
        KIXSVWSRIIWQPPSPSYWSWRSIRSWSGRIVG
        RIVSSTZQFMXPPSPFRGWEQEIEVSYYMGWPHSPS
        GIEEMVXHVQPMSEEMSXQRMVXIIUM
        KPYESMVXPMIMEHHVSTYKMSISDESGQSEQUGWVEW
        WWEXDETRWMYEXRRQMIWFVSWEVIIFSIS
        ISYTEQGSPIRKIVMQSMSYYWREGIKIISIVPVCP
        WMWRCSHMGMFSWERIIVVGSXZQS
        \end{verbatim}

        \begin{verbatim}
        [3.] IAXLBOZSZODIRBILBOPBPOXRQBPQMXPOOOP
        JIFILXPBPKCXLJMBOPXPNAKZIIRKBZBXR
        YKXNPLBLRKXBLPLXLOBYBZONXJBIQXQBBBE
        AFXIJGVBLOXXXQJILZLFBLOBKJRL
        OXXRYQXXIQIKPVXFZXFJBIZLLILFBPOLRFBSO
        QZLXXEAZLIZVAOLCFQUAOFABIFGKOQII
        XXXBPBLAPPARPCZFSXFBPBBPIKRPIAPBIXB
        XMFMXPQBALAXPNOQOPIBRKLLBP
        \end{verbatim}

        \begin{verbatim}
        [4.] IMBJAMIWWMCQZTXLAMLAOMKFQVWWWJQIIICM
        XBWAUKTMKQAZIWTWKUACWTIWDKTYIVZMC
        MOCTLZZAUUAAWAVBQUZHQQCOCVQCVWGCUM
        MMKVWWIZMWZAAZIWAWCASMWNBWMVI
        LTTZMYZQCTBWAMBBHCMRMZUOWTIFMMCMMXII
        ZIVYDIMQAWIIIWXIAWQMIBMAJVQAWIIWK
        TRZWRVMMBWQKCWKQLADTARQIUIXCMKMMJVA
        MVWTCMTIVWAKCBMUBIOIAZZF
        \end{verbatim}

        \begin{verbatim}
        [5.] CUKTEUPTORPVCGTGECGCWBKEGNPSGCPFONP
        PQWUQQCCNKPVQUTQUCQQGPQOUKQGWJEPP
        UUOGGGNQEGGKCUSFTVKGSQFKOAQDNGTGPC
        TNPGWAUGONREDJQTUGPUOTNGTQUGQF
        GFVCTWGDNCQNKNQCJPPQEKGGDKFKOEPUTTOG
        QTTWKEUQQUPERNWNGTECHQGJGECWUFUIW
        NGOEQENGWGGTGRWVGAKKWQFTKLCN
        NTUTKUGTVTKNQCUXGGWGKUKCNWNWGKK
        \end{verbatim}

        \begin{verbatim}
        [6.] HECERMOIEOOEQIILRLJBTQOLROAUSGE
        OUEFTELASRLMSOOESITSONRNAASBQDRSEAOOT
        CTEHIMOSADNGMSUOOOLUUEAUEPDRAKIN
        HYRADRELESIHAAEAPCILTAOOHNAPVMLA
        SAIDAEERABDENERVEFTOEOZNRCOCAESEREA
        LYARIOECNTMOEOAETSEALASLUROQSHOIIE
        ENIIRECSDLNIBAAEGLEBNDEALENTQIGMNU
        STEESIRMEENERIDDNCENESSTC
        \end{verbatim}

        \newpage
        \begin{verbatim}
        [7.] GGSEAWKLFJKJMVEIAYSAAMWSJHDWAAKKW
        SJGESEDGDGAEWEGFSSKDGDTFIAMSSVFFFWJ
        SWFWVGKGFALMGGWFWJAYWFZFFWWGVZSV
        WSWZSUNWFLFAJJKYSSFSWWUEAWYGWAWT
        AVEWLUKGVWWVVKWASJGSFDLWWWWSSUWWS
        KKJEDAGHMJVJWKJJVVSUKFYNMDYLFMDSIESJ
        FGLSWHMLAKLLSJFGSGFJEWSKWWGAM
        TMGKKVGJDATAWZJEFKEGWSZKLKSXGS
        \end{verbatim}
        
        Ahora, realizaremos un análisis de frecuencia de cada subcriptograma:
        \begin{enumerate}
            \item Tenemos que
            \begin{center}
                \begin{tabular}{|c|c|}
                \hline
                A & $31-11\%$ \\ \hline
                B & $7-2\%$ \\ \hline
                C & $3-1\%$ \\ \hline
                D & $20-7\%$ \\ \hline
                E & $22-8\%$ \\ \hline
                F & $12-4\%$ \\ \hline
                G & $7-2\%$ \\ \hline
                H & $1-0.3\%$ \\ \hline
                I & $0-0\%$ \\ \hline
                J & $2-0.7\%$ \\ \hline
                K & $3-1\%$ \\ \hline
                L & $2-0.7\%$ \\ \hline
                M & $41-15\%$ \\ \hline
                N & $5-2\%$ \\ \hline
                O & $4-1\%$ \\ \hline
                P & $12-4\%$\\ \hline
                Q & $26-10\%$ \\ \hline
                R & $3-1\%$ \\ \hline
                S & $4-1\%$ \\ \hline
                T & $1-0.3\%$ \\ \hline
                U & $13-4\%$ \\ \hline
                V & $0-0\%$ \\ \hline
                W & $1-0.3\%$ \\ \hline
                X & $15-5\%$ \\ \hline
                Y & $11-4\%$ \\ \hline
                Z & $16-6\%$ \\ \hline
                \end{tabular}
            \end{center}

            \item Tenemos que
            \begin{center}
                \begin{tabular}{|c|c|}
                \hline
                A & $0-0\%$ \\ \hline
                B & $0-0\%$ \\ \hline
                C & $2-0.7\%$ \\ \hline
                D & $3-1.1\%$ \\ \hline
                E & $23-9\%$ \\ \hline
                F & $5-2\%$ \\ \hline
                G & $11-4\%$ \\ \hline
                H & $6-2\%$ \\ \hline
                I & $31-11\%$ \\ \hline
                J & $0-0\% $ \\ \hline
                \end{tabular}
            \end{center} 

            \begin{center}
                \begin{tabular}{|c|c|}
                \hline
                K & $5-2\%$ \\ \hline
                L & $1-0.3\%$ \\ \hline
                M & $20-7\%$ \\ \hline
                N & $0-0\%$ \\ \hline
                O & $0-0\%$ \\ \hline
                P & $16-6\%$\\ \hline
                Q & $12-4\%$ \\ \hline
                R & $15-5\%$ \\ \hline
                S & $39-15\%$ \\ \hline
                T & $4-1\%$ \\ \hline
                U & $2-0.7\%$ \\ \hline
                V & $20-7\%$ \\ \hline
                W & $21-8\%$ \\ \hline
                X & $12-4\%$ \\ \hline
                Y & $11-4\%$ \\ \hline
                Z & $2-0.7\%$ \\ \hline
                \end{tabular}
            \end{center}

            \item Tenemos que
            \begin{center}
                \begin{tabular}{|c|c|}
                \hline
                A & $31-12\%$ \\ \hline
                B & $7-2\%$ \\ \hline
                C & $3-1\%$ \\ \hline
                D & $20-7\%$ \\ \hline
                E & $22-8\%$ \\ \hline
                F & $12-4\%$ \\ \hline
                G & $7-2\%$ \\ \hline
                H & $1-0.3\%$ \\ \hline
                I & $0-0\%$ \\ \hline
                J & $2-0.7\%$ \\ \hline
                K & $3-1\%$ \\ \hline
                L & $2-0\%$ \\ \hline
                M & $41-15\%$ \\ \hline
                N & $5-2\%$ \\ \hline
                O & $4-1\%$ \\ \hline
                P & $12-4\%$\\ \hline
                Q & $26-10\%$ \\ \hline
                R & $3-1\%$ \\ \hline
                S & $4-1\%$ \\ \hline
                T & $1-0.3\%$ \\ \hline
                U & $12-4.5\%$ \\ \hline
                V & $0-0\%$ \\ \hline
                W & $1-0.3\%$ \\ \hline
                X & $15-6\%$ \\ \hline
                Y & $11-4\%$ \\ \hline
                Z & $16-6\%$ \\ \hline
                \end{tabular}
            \end{center}

            \newpage
            \item Tenemos que 
            \begin{center}
                \begin{tabular}{|c|c|}
                \hline
                A & $24-9\%$ \\ \hline
                B & $11-4\%$ \\ \hline
                C & $17-6\%$ \\ \hline
                D & $3-1\%$ \\ \hline
                E & $0-0\%$ \\ \hline
                F & $3-1\%$ \\ \hline
                G & $1-0.3\%$ \\ \hline
                H & $2-0.7\%$ \\ \hline
                I & $28-11\%$ \\ \hline
                J & $4-1\%$ \\ \hline
                K & $11-4\%$ \\ \hline
                L & $5-2\%$ \\ \hline
                M & $34-13\%$ \\ \hline
                N & $1-0.3\%$ \\ \hline
                O & $5-2\%$ \\ \hline
                P & $0-0\%$\\ \hline
                Q & $15-5\%$ \\ \hline
                R & $4-1\%$ \\ \hline
                S & $1-0.3\%$ \\ \hline
                T & $14-5\%$ \\ \hline
                U & $9-3\%$ \\ \hline
                V & $13-5\%$ \\ \hline
                W & $31-12\%$ \\ \hline
                X & $5-2\%$ \\ \hline
                Y & $3-1\%$ \\ \hline
                Z & $16-6\%$ \\ \hline
                \end{tabular}
            \end{center}

            \item Tenemos que 
            \begin{center}
                \begin{tabular}{|c|c|}
                \hline
                A & $3-1\%$ \\ \hline
                B & $1-0.3\%$ \\ \hline
                C & $18-7\%$ \\ \hline
                D & $4-1\%$ \\ \hline
                E & $14-5\%$ \\ \hline
                F & $8-3\%$ \\ \hline
                G & $39-15\%$ \\ \hline
                H & $1-0.3\%$ \\ \hline
                I & $1-0.3\%$ \\ \hline
                J & $4-1\%$ \\ \hline
                K & $21-8\%$ \\ \hline
                L & $1-0.3\%$ \\ \hline
                M & $0-0\%$ \\ \hline
                N & $20-7\%$ \\ \hline
                O & $10-4\%$ \\ \hline
                P & $17-6\%$\\ \hline
                Q & $25-9\%$ \\ \hline
                R & $4-1\%$ \\ \hline
                S & $3-1\%$ \\ \hline
                T & $22-8\%$ \\ \hline
                \end{tabular}
            \end{center}

            \begin{center}
                \begin{tabular}{|c|c|}
                    \hline
                    U & $22-8\%$ \\ \hline
                    V & $6-2\%$ \\ \hline
                    W & $15-6\%$ \\ \hline
                    X & $1-0.3\%$ \\ \hline
                    Y & $0-0\%$ \\ \hline
                    Z & $0-0\%$ \\ \hline
                \end{tabular}
            \end{center}

            \item Tenemos que 
            \begin{center}
                \begin{tabular}{|c|c|}
                \hline
                A & $27-10\%$ \\ \hline
                B & $5-2\%$ \\ \hline
                C & $9-3\%$ \\ \hline
                D & $10-3\%$ \\ \hline
                E & $42-16\%$ \\ \hline
                F & $2-0.7\%$ \\ \hline
                G & $4-1\%$ \\ \hline
                H & $6-2\%$ \\ \hline
                I & $18-6\%$ \\ \hline
                J & $1-0.3\%$ \\ \hline
                K & $1-0.3\%$ \\ \hline
                L & $15-5\%$ \\ \hline
                M & $8-3\%$ \\ \hline
                N & $16-6\%$ \\ \hline
                O & $25-9\%$ \\ \hline
                P & $3-1\%$\\ \hline
                Q & $5-1\%$ \\ \hline
                R & $18-7\%$ \\ \hline
                S & $20-7\%$ \\ \hline
                T & $12-4\%$ \\ \hline
                U & $8-3\%$ \\ \hline
                V & $2-0.7\%$ \\ \hline
                W & $0-0\%$ \\ \hline
                X & $0-0\%$ \\ \hline
                Y & $2-0.7\%$ \\ \hline
                Z & $1-0.3\%$ \\ \hline
                \end{tabular}
            \end{center}
            
            \item Tenemos que 
            \begin{center}
                \begin{tabular}{|c|c|}
                \hline
                A & $20-7\%$ \\ \hline
                B & $0-0\%$ \\ \hline
                C & $0-0\%$ \\ \hline
                D & $10-4\%$ \\ \hline
                E & $13-5\%$ \\ \hline
                F & $22-8\%$ \\ \hline
                G & $24-9\%$ \\ \hline
                H & $3-1\%$ \\ \hline
                I & $3-1\%$ \\ \hline
                J & $19-7\%$ \\ \hline
                K & $20-7\%$ \\ \hline
                L & $11-4\%$ \\ \hline
                M & $10-4\%$ \\ \hline
                N & $2-0.7\%$ \\ \hline
                \end{tabular}
            \end{center}

            \begin{center}
                \begin{tabular}{|c|c|}
                    \hline
                    O & $0-0\%$ \\ \hline
                    P & $0-0\%$\\ \hline
                    Q & $0-0\%$ \\ \hline
                    R & $0-0\%$ \\ \hline
                    S & $31-12\%$ \\ \hline
                    T & $4-1\%$ \\ \hline
                    U & $5-2\%$ \\ \hline
                    V & $13-5\%$ \\ \hline
                    W & $38-14\%$ \\ \hline
                    X & $1-0.3\%$ \\ \hline
                    Y & $6-2.3\%$ \\ \hline
                    Z & $5-2\%$ \\ \hline
                \end{tabular}
            \end{center}
        \end{enumerate}

        Realizando un análisis de cada fragmento obtenemos:
        \begin{itemize}
            \item De 1, vemos que la letra con mayor frecuencia es M.

            \item De 2, la letra con mayor frecuencia es S, pero MS no tendía 
            mucho sentido, entonces tomamos la vocal con mayor frecuencia, que 
            es E.

            \item De 3, la mayor frecuencia sobre una letra se presenta en M.
            
            \item De 4, vemos que la letra con mayor frecuencia es M, pero 
            necesitamos una vocal, y la que tiene mayor frecuencia es I.

            \item De 5, observamos que la letra con mayor frecuencia es G.

            \item De 6, vemos que la letra con mayor frecuencia es A.

            \item De 7, obervamos que la letra con mayor frecuencia es W, pero 
            ninguna palabra termina con W, entonces tomamos la segunda, que es 
            S.
        \end{itemize}
        
        Observemos como la palabra generada es MEMIGAS, pero eso no tiene mucho 
        sentido, entonces sustituiremos la tercera y quinta letra de la palabra.
        Al hacer cambios sobre estas dos letras, descubrimos que la palabra clave 
        es MEXICAS.
        
        ¿Por qué la palabra clave no es medidas? Esto se da ya que en la quinta
        posición, esa misma letra tiene una frecuencia muy baja.

        % Ejercicio 5.b
        \item Dar la clave de cifrado.
        
        \textsc{Solución:} MEXICAS.
        
        % Ejercicio 5.c
        \item Descifrar el mensaje.
        
        \textsc{Solución:}
        \begin{verbatim}
        HOLA AHORA DESEO PLATICAR SOBRE MIS ESCRITORES MEXICANOS FAVORITOS 
        COMENZARE POR ALGUNOS DE LITERATURA QUE HE LEIDO EL PRIMERO DEL QUE 
        ESCRIBIRE ALGO ES DE JAIME SABINES GUTIERREZ QUE NACIO EN TUXCLA    
        GUTIERREZ EN LO PERSONAL NO TOQUE SU POESIA DABA GIROS INESPERADOS 
        PARA MUESTRA LEAMOS UN FRAGMENTO DEL POEMA TITULADO LOS AMOROSOS 
        LOS AMOROSOS CALLAN EL AMOR ES EL SILENCIO MAS FINO EL MAS 
        TEMBLOROSO EL MAS INSOPORTABLE LOS AMOROSOS BUSCAN LOS AMOROSOS SON 
        LOS QUE ABANDONAN SON OS QUE CAMBIAN LOS QUE OLVIDAN SU CORAZON LES 
        DICE QUE NUNCA HAN DE ENCONTRAR NO ENCUENTRAN BUSCAN EN ESTE 
        FRAGMENTO QUE HEMOS LEIDO PODEMOS VERLOS AMOROSOS BUSCAN E   
        INMEDIATAMENTE LE SIGUE LOS AMOROSOS SON LOS QUE ABANDONAN OTRO 
        ESCRITOR ES EMILIO ABREU GOMEZ QUE NACIO EN MERIDA HE AQUI UN     
        FRAGMENTO MUY PEQUENO DE SU LIBRO TITULADO CANEK HISTORIA Y 
        LEYENDA DE UN HEROE MAYA EL HERRERO DE LA HACIENDA SE ACERCO AL 
        NUEVO AMO Y LE DIJO SENOR YA ESTA TERMINADO EL HIERRO PARA MARCAR 
        A LAS BESTIAS HAGO OTRO PARA MARCAR A LOS INDIOS  EL AMO CONTESTO    
        USA EL MISMO CANEK ROMPIO EL HIERRO. EN ESTE FRAGMENTO PODEMOS  
        VER QUE EMILIO NO LE AGRADABA LA DESIGUALDAD LA ULTIMA OBRA 
        DE LITERATURA QUE CITARE ES EL LIBRO TITULADO EL LABERINTO DE LA 
        SOLEDAD Y SIN DUDA EL ESCRITOR ES OCTAVIO PAZ HE AQUI UN 
        FRAGMENTO VIEJO O ADOLECENTE CRIOLLO O MEZTIZO GENERAL OBRERO O     
        LICENCIADO EL MEXICANO SE ME APARECE COMO UN SER QUE SE ENCIERRA Y 
        SE PRESERVA MASCARA EL ROSTRO Y MASCARA LA SONRISA AQUI OCTAVIO PAZ 
        HACE UNA DESCRIPCION DE NOSOTROS LOS MEXICANOS MUY ACERTADA POR 
        OTRO LADO NO PUEDEN FALTAR MIS ESCRITORES MEXICANOS DE ALGEBRA 
        FAVORITOS UNO DE ELLOS ES HUGO ALBERTO RINCON MEJIA QUE EN SUS 
        LIBROS HA NOTADO QUE LA SIMBOLOGIA DE ACUERDO AL LENGUAJE NOS 
        PERMITE ASOCIAR MEJOR EL CONCEPTO DEL CUAL SE ESTA ESTUDIANDO EL 
        SIGUIENTE ESCRITOR FUE BASICO PARA MI CUANDO VI TEORIA DE GALOIS
        Y LO QUE VI EN SUS LIBROS ES UN MANEJO DE LAS IDEAS CLARAS Y EN 
        MI LENGUAJE EL ESPANOL. EL ULTIMO DEL QUE ESCRIBIRE ES GUILLERMO 
        GRABINSKY EN SUS CLASES DESPERTO MI INTERES POR EL ANALISIS Y
        SU LIBRO TEORIA DE LA MEDIDA SE HA CONVERTIDO EN MI BASE EN LOS  
        CURSOS QUE IMPARTIDO ANTES DE TERMINAR ESTA CHARLA LES PREGUNTO 
        CUALES SON SUS AUTORES FAVORITOS MEXICANOS.
        \end{verbatim}
    \end{enumerate}
    
    % Ejercicio 6.
    \item Dado el siguiente mensaje cifrado con Hill, del cual se tiene que 
    \begin{center}
        IQ SU NF WI FE IY IK CC KO IG UV
    \end{center}
    
    proviene de 
    \begin{center}
        Como ho ye nd ia es mu yc om un
    \end{center}
    
    \begin{verbatim}
    IQ SU BX EW AF NB CN OD IU BV CG YI OD NF WI FE IY IK CC KO IG UV VD
    NB RY BZ EZ YI EL UQ WR IY DG MR NU YY RZ MK KT OH SF AB MW OI ZU
    SF US IB EY AO XI CN LB DN EZ CN OT KY XI CA LB XO NB KY LP WD OI YU
    HV OM NN AP EM CC KO LH RZ FL IK LH FP IM HJ AN SO MV VD LB CC SL
    OQ DF IK RG MU YB PN RZ LH WA SG QK EZ RT KT SO PN WK LH LP EZ RT
    YM RR TX KT AN UV WC BX UQ IO VL GS CG OE DF LH NH XJ BX EJ GM BU
    AN DF IK RG KS XI CN IK MN QS VC CO SE SW IK IL RT RM OH TX YM RG SO
    YM SU GV SO GO WD QG AP CO SO GO OH AN ZP BX QQ UQ SO SC ZR VX UN
    PN SF RT MK GO EZ SO IL RT PN SC VD FS AK UI LU SO DJ UV PT TX RZ MK
    CC KO LH XJ BX RR GO KJ YG CG SU QQ BG WP AP QS CX WD IK GO UR AN
    BX YI ZL JT IK TF PN ZH GM ZH IY KT SF MU IG FL TF YI OD YU HV ZE AN
    TX TF JT GO CA CX WC JC CJ GC RZ DJ VD BU GH SO LH JT EJ SW EW OD
    RR SC VL ZH IK RT CT CJ YY US SO NF KB PN QQ XI IK LH EW IG FL TF JT
    RZ WA OD VH FE UQ EZ TF JV QQ LB ZA SO HV YU VL ZH IK EZ XF YL QG
    TX RR CU MU IG AK PN YB RT LH KT YC AN CN GX OI RR CU CG VD LP GH
    SF MU OI ZH IG FL TF YI OD OT PT RZ ZL QQ TX KT QT XJ RT IL EY GO IK
    OD YG SR EY PT RZ DF AB SO KY ZL CG KY XI SO ME FX GH SR GH SF EG
    TX MK PT RZ ZL QQ TX BX VL SY NN IH OG TF SH GM NN VL ZH IK EZ BR
    OE CG IQ LB FX GH SF SR GH IX EZ IG EZ SO BR OE IQ GI EY UF GH CN IK
    JX ZL YB RR GI IQ YR PN FE CN RR EJ GH NB UQ SO KY ZH OI NN LP GC HV
    SC PN EY JT FE IK EM DM US OT MU YB PN RZ GO UR AN SF OU AP CW OT
    MC CO VD BU AN RT GO WD KJ EZ NH RL YB EZ IK RT SU KY XI ME EW UI
    PF CN EY YG SG QK EZ RT PT YM
    \end{verbatim}
    
    \begin{enumerate}
        % Ejercicio 6.a
        \item Encontrar la matriz de cifrado planteando con cuáles congruencias
        se obtiene.
        
        \textsc{Solución:} Primero, asociamos cada letra de nuestro alfabeto
        (supondremos que es de $26$ letras, pues quitamos la ñ) con un número,
        de la forma $(a \rightarrow 0, ..., z \rightarrow 25)$. Como en la
        correspondencia anterior solamente aparecen $26$ letras, entonces hay 
        que trabajar con los números enteros \textbf{módulo 26}.
        
        Ahora, con la información que nos dan como premisa, tenemos las 
        siguientes correspondencias:
        \begin{equation*}
            \begin{bmatrix} I  \\ Q \end{bmatrix} =
            \begin{bmatrix} 8 \\ 16 \end{bmatrix} \mapsto
            \begin{bmatrix} 2 \\ 14 \end{bmatrix} =
            \begin{bmatrix} C \\ O \end{bmatrix}
        \end{equation*}
        \begin{equation*}
            \begin{bmatrix} S  \\ U \end{bmatrix} =
            \begin{bmatrix} 18 \\ 20 \end{bmatrix} \mapsto
            \begin{bmatrix} 12 \\ 14 \end{bmatrix} =
            \begin{bmatrix} M \\ O \end{bmatrix}
        \end{equation*}
        \begin{equation*}
            \begin{bmatrix} N \\ F \end{bmatrix} =
            \begin{bmatrix} 13 \\ 5 \end{bmatrix} \mapsto
            \begin{bmatrix} 7 \\ 14 \end{bmatrix} =
            \begin{bmatrix} H \\ O \end{bmatrix}
        \end{equation*}
        \begin{equation*}
            \begin{bmatrix} W  \\ I \end{bmatrix} =
            \begin{bmatrix} 22 \\ 8 \end{bmatrix} \mapsto
            \begin{bmatrix} 24 \\ 4 \end{bmatrix} =
            \begin{bmatrix} Y \\ E \end{bmatrix}
        \end{equation*}
        \begin{equation*}
            \begin{bmatrix} F \\ E \end{bmatrix} =
            \begin{bmatrix} 5 \\ 4 \end{bmatrix} \mapsto
            \begin{bmatrix} 13 \\ 3 \end{bmatrix} =
            \begin{bmatrix} N \\ D \end{bmatrix}
        \end{equation*}
        \begin{equation*}
            \begin{bmatrix} I \\ Y \end{bmatrix} =
            \begin{bmatrix} 8 \\ 24 \end{bmatrix} \mapsto
            \begin{bmatrix} 8 \\ 0 \end{bmatrix} =
            \begin{bmatrix} I \\ A \end{bmatrix}
        \end{equation*}
        \begin{equation*}
            \begin{bmatrix} I \\ K \end{bmatrix} =
            \begin{bmatrix} 8 \\ 10 \end{bmatrix} \mapsto
            \begin{bmatrix} 4 \\ 18 \end{bmatrix} =
            \begin{bmatrix} E \\ S \end{bmatrix}
        \end{equation*}
        \begin{equation*}
            \begin{bmatrix} C \\ C \end{bmatrix} =
            \begin{bmatrix} 2 \\ 2 \end{bmatrix} \mapsto
            \begin{bmatrix} 12 \\ 20 \end{bmatrix} =
            \begin{bmatrix} M \\ U \end{bmatrix}
        \end{equation*}
        \begin{equation*}
            \begin{bmatrix} K \\ O \end{bmatrix} =
            \begin{bmatrix} 10 \\ 14 \end{bmatrix} \mapsto
            \begin{bmatrix} 24 \\ 2 \end{bmatrix} =
            \begin{bmatrix} Y \\ C \end{bmatrix}
        \end{equation*}
        \begin{equation*}
            \begin{bmatrix} I \\ G \end{bmatrix} =
            \begin{bmatrix} 8 \\ 6 \end{bmatrix} \mapsto
            \begin{bmatrix} 14 \\ 12 \end{bmatrix} =
            \begin{bmatrix} O \\ M \end{bmatrix}
        \end{equation*}
        \begin{equation*}
            \begin{bmatrix} U \\ V \end{bmatrix} =
            \begin{bmatrix} 20 \\ 21 \end{bmatrix} \mapsto
            \begin{bmatrix} 20 \\ 13 \end{bmatrix} =
            \begin{bmatrix} U \\ N \end{bmatrix}
        \end{equation*}
        
        Recordemos que buscamos la matriz
        \begin{equation*}
            \begin{bmatrix} a & b \\ c & d \end{bmatrix}  \pmod{26}
        \end{equation*}

        que hace las transformaciones indicadas arriba sean ciertas. Usando
        la transformación $6$ tenemos que
        \begin{equation*}
            \begin{bmatrix} a & b \\ c & d \end{bmatrix}
            \begin{bmatrix} 8 \\ 0 \end{bmatrix} \equiv
            \begin{bmatrix} 8 \\ 24 \end{bmatrix} \pmod{26}
        \end{equation*}
        
        de donde obtenemos
        \begin{equation*}
            8a + 0b \equiv 8 \pmod{26} \Rightarrow 4a \equiv 4 \pmod{13}
        \end{equation*}
        \begin{equation*}
            8c + 0d \equiv 24 \pmod{26} \Rightarrow 4c \equiv 12 \pmod{13}
        \end{equation*}

        las cuales tienen el mismo conjunto de soluciones que
        \begin{equation*}
            a \equiv 1 \pmod{13}
        \end{equation*}
        \begin{equation*}
            c \equiv 3 \pmod{13}
        \end{equation*}
        
        Obteniéndo como soluciones
        \begin{align}
            a = 1 + 13k, k \in \mathbb{Z} \\
            c = 3 + 13l, l \in \mathbb{Z}
        \end{align}
        
        Ahora bien, usando la primer transformación tenemos que 
        \begin{equation*}
            \begin{bmatrix} a & b \\ c & d \end{bmatrix}
            \begin{bmatrix} 2 \\ 14 \end{bmatrix} \equiv
            \begin{bmatrix} 8 \\ 16 \end{bmatrix} \pmod{26}
        \end{equation*}
        
        de donde obtenemos
        \begin{equation*}
            2a + 14b \equiv 8 \pmod{26} \Rightarrow a + 7b \equiv 7 \pmod{13}
        \end{equation*}
        \begin{equation*}
            2c + 14d \equiv 16 \pmod{26} \Rightarrow c + 7c \equiv 6 \pmod{13}
        \end{equation*}

        Sustituyendo $(19)$ y $(20)$ en las ecuaciones anteriores tenemos que 
        \begin{align*}
            (1 + 13k) + 7b \equiv q \pmod{13}, k \in \mathbb{Z} \\
            (3 + 13l) + 7c \equiv 3 + 13l, l \in \mathbb{Z}
        \end{align*}
        
        de donde obtenemos 
        \begin{align*}
            7b \equiv 3 \pmod{13} \\
            7d \equiv 5 \pmod{13}
        \end{align*}
        
        concluyendo que 
        \begin{align*}
            b = 6 + 13m, m \in \mathbb{Z} \\
            d = 10 + 13n, n \in \mathbb{Z}
        \end{align*}
        
        En particular, si $k = l = m = n = 1$ obtenemos que $a = 14$, $b = 19$,
        $c = 3$ y $d = 23$. Por lo tanto, la matriz que estamos buscando es
        \begin{equation*}
            \begin{bmatrix} 14 & 19 \\ 3 & 23\end{bmatrix} 
        \end{equation*}

        % Ejercicio 6.b
        \item Calcular la matriz inversa paso a paso.
        
        \textsc{Solución:} Primero, calculamos el determinante de nuestra 
        matriz.
        \begin{align*}
            \det \left(\begin{bmatrix} 14 & -3 \\ -19 & 23\end{bmatrix}\right)
            &= 23 \cdot 14 - (-19 \cdot -3) \\
            &= 322 - 57 \\
            &= 265
        \end{align*}
        
        Luego, calculamos la matriz adjunta
        \begin{align*}
            Ad \left(\begin{bmatrix} 14 & -3 \\ -19 & 23\end{bmatrix}\right)
            &= \begin{bmatrix} 23 & -3 \\ -19 & 14\end{bmatrix}
        \end{align*}
        
        Por lo tanto, la matriz inversa es igual a 

        \begin{align*}
            \begin{bmatrix} 14 & 19 \\ 3 & 23\end{bmatrix} 
            &= \frac{1}{265} \begin{bmatrix} 23 & -3 \\ -19 & 14\end{bmatrix} \\
            &= \begin{bmatrix} 15 & 17 \\ 15 & 8 \end{bmatrix} \pmod{26}
        \end{align*}
        
        % Ejercicio 6.c
        \item Descifrar el mensaje.
        
        \textsc{Solución:} Utilizamos el código de nuestro proyecto 1 para
        poder descifrar el mensaje. 
        \begin{verbatim}
        COMO DESCUBRIERON SU VOCACION HOY EN DIA ES MUY COMUN POR 
        INFLUENCIA FAMILIAR POR PROGRAMAS DE TELEVISION PELICULAS E 
        INTERNET GENERALMENTE EN EL PRIMER MEDIO YA HAY UNA IDEA MUY CLARA 
        PUES LA FAMILIA NOS EXPONE MUCHO A NUESTRA CARRERA LAS SIGUIENTES 
        DOS REGULARMENTE SOLO NOS DAN UNA IDEA MUY VAGA CASI NULA PERO    
        DESPIERTAN NUESTRO INTERES Y LA ULTIMA QUE ES INTERNET NOSOTROS 
        SOMOS LOS QUE DECIDIMOS QUE TANTO DESEAMOS SABER SOBRE EL TEMA QUE 
        NOS INTERESA POR LO CUAL NOS DA UN PANORAMA MUY CLARO DE LO QUE       
        BUSCAMOS EN LA VIDA USTED ES QUE HAN DECIDIDO ESTAR EN CIENCIAS DE 
        LA COMPUTACION YA HABRAN NOTADO QUE ESTA INVOLUCRADA POR TODOS 
        LADOS PUES CON LOS AVANCES TECNOLOGICOS HOY PRESENTES LAS 
        COMPUTADORAS SON FUNDAMENTALES EN ELLOS HAY AVANCES EN 
        BIOTECNOLOGIA COMO CREAR TELAS DE MANERA BIOLOGICA POR MODELACION 
        COMPUTACIONAL PARA DISENOS DE PROTEINAS QUE SON USADAS PARA 
        NUEVOS MEDICAMENTOS O EL MODELO DEL GENOMA PARA DISENO DE VACUNAS 
        HAY TAMBIEN AVANCES EN FISICA CON EL MODELADO DE FENOMENOS FISICOS 
        Y ASI PODER ESTUDIARLOS Y COMPRENDERLOS PODRIAMOS MENCIONAR MUCHAS 
        AREAS DONDE SE APLICA LA CARRERA QUE HAN ELEGIDO Y ALGO IMPORTANTE 
        QUE DEBEN PENSAR EN ESTE MOMENTO ES CUAL SERA SU SIGUIENTE PASO.
        \end{verbatim}
    \end{enumerate}
\end{enumerate}
\end{document}
