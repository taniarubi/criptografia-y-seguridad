\documentclass[letterpaper,11pt]{article}

% Soporte para los acentos.
\usepackage[utf8]{inputenc}
\usepackage[T1]{fontenc}
% Idioma español.
\usepackage[spanish,mexico, es-tabla]{babel}
% Soporte de símbolos adicionales (matemáticas)
\usepackage{multirow}
\usepackage{amsmath}
\usepackage{amssymb}
\usepackage{amsthm}
\usepackage{amsfonts}
\usepackage{mathtools}
\DeclarePairedDelimiter\floor{\lfloor}{\rfloor}
\usepackage{latexsym}
\usepackage{enumerate}
\usepackage{ragged2e}
\usepackage{listings}
\usepackage{xcolor}
\usepackage{array}
% Modificamos los márgenes del documento.                                       %
\usepackage[lmargin=2cm,rmargin=2cm,top=2cm,bottom=2cm]{geometry}

\definecolor{codegreen}{rgb}{0,0.6,0}
\definecolor{codegray}{rgb}{0.5,0.5,0.5}
\definecolor{codepurple}{rgb}{0.58,0,0.82}
\definecolor{backcolour}{rgb}{0.95,0.95,0.92}

\lstdefinestyle{mystyle}{
    backgroundcolor=\color{backcolour},   
    commentstyle=\color{codegreen},
    keywordstyle=\color{magenta},
    numberstyle=\tiny\color{codegray},
    stringstyle=\color{codepurple},
    basicstyle=\ttfamily\footnotesize,
    breakatwhitespace=false,         
    breaklines=true,                 
    captionpos=b,                    
    keepspaces=true,                 
    numbers=left,                    
    numbersep=5pt,                  
    showspaces=false,                
    showstringspaces=false,
    showtabs=false,                  
    tabsize=2
}

\lstset{style=mystyle}

\title{Criptografía y Seguridad \\ Tarea 2}
\author{Altamirano Vázquez Jesús Fernando \\
        Rubí Rojas Tania Michelle}
\date{\today}

\begin{document}
\maketitle

\begin{enumerate}
    % Ejercicio 1.                                                             
     \item Mediante el algoritmo de de ro-Pollar descomponer el número 
    $n = 557437$.
    
    \textsc{Solución:} Para realizar la descomposición, haremos lo siguiente:
    \begin{enumerate}
        \item Asignamos $a = 2$ y $b = 2$.
        \item Para $i = 1,2,...$ 
        \begin{itemize}
            \item Se hace $a = a^{2} + 1 \pmod{n}$, $b = b^{2} + 1 \pmod{n}$ y 
            $b = b^{2} + 1 \pmod{n}$. 
            \item Calculamos $(a - b, n)$.
            \item Si $(a - b, n) > 1$ se tiene que hemos encontrado un factor 
            no trivial de $n$.
            \item Si $(a - b, n) = 1$ o $(a - b, n) = n$ avanza a la siguiente 
            $i$.
        \end{itemize}
    \end{enumerate}
    
    Como $n = 557437$ entonces obtenemos lo siguiente:
    \begin{center}
        \begin{tabular}{|c|c|c|c|} 
            \hline
            $i$ & $x_{i}$ & $b = f(x_{i})$ & $(a - b, n)$ \\ \hline
            1 & 5 & 26 & 1 \\ \hline
            2 & 26 & 458,330 & 1 \\ \hline
            3 & 677 & 108,979 & 1 \\ \hline
            4 & 458,330 & 31,871 & 1 \\ \hline
            5 & 157,510 & 408,810 & 1 \\ \hline
            6 & 108,979 & 200,153 & 1 \\ \hline
            7 & 221,157 & 96,572 & 1 \\ \hline
            8 & 31,871 & 45,835 & 1 \\ \hline
            9 & 110,428 & 130,880 & 1 \\ \hline
            10 & 408,810 & 176,439 & 1 \\ \hline
            11 & 429,131 & 481,892 & 1 \\ \hline
            12 & 200,153 & 225,945 & 1 \\ \hline
            13 & 455,968 & 226,069 & 389 \\ \hline
        \end{tabular}
    \end{center}
    
    Por lo tanto, la descomposición de $n$ sería de la forma $n = (389)(1433)$.
    
    % Ejercicio 2.
    \item Sea $\mathbb{Z}_{10007}$ y $S = \{2, 3, 5, 7\}$. Calcular el índice 
    de $\beta = 9451$ en $\alpha = 5$.
    \begin{enumerate}
        % Ejercicio 2.a
        \item Mostrar que $\alpha$ es raíz primitiva de $\mathbb{Z}^{*}_{10007}$.

        \textsc{Solución:} Sabemos que un entero positivo $\alpha$ es una raíz 
        primitiva módulo $n$ si $(\alpha, n) = 1$ y $ord(\alpha) = \varphi(n)$,
        siendo $\varphi$ la \textit{función phi de Euler}.
        La \textit{función phi de Euler} se define de la siguiente forma:
        \begin{equation*}
            \varphi(m) = |\{n \in \mathbb{N} | n \leq m \land (m, n) = 1 \}|
        \end{equation*}

        Es decir, si $n$ es un entero positivo, entonces $\varphi(n)$ se 
        define como el número de enteros positivos menores o iguales a $n$ y 
        coprimos con $n$. 
        
        Dada la siguiente implementación de la función Euler
        \begin{lstlisting}[language=Python]
            # Regresa el minimo comun divisor de dos numeros. 
            def mcd(a, b):
                if (b == 0):
                    return a 
                else:
                    return mcd(b, a % b)
            
            '''
            La funcion phi de Euler regresa el numero de enteros positivos que 
            son menores o iguales a n y que ademas son primos relativos con n. 
            '''
            def phi_Euler(n):
                relativos = []
                for i in range (1, n):
                    if (mcd(n, i) == 1):
                        relativos.append(i)
                return len(relativos)
            
            def main():
                print (phi_Euler(10007))
            
            if __name__ == "__main__":
                main()
        \end{lstlisting}
        
        obtenemos que $\varphi(10007) = 10006$ (esto es cierto ya que $n = 10007$ 
        es un primo, y por lo tanto $\varphi(10007) = 10007 - 1 = 10006$). 
        
        El órden de un elemento $x$ de un grupo $G$ es el más pequeño entero 
        positivo $i$ tal que $x^{i} = e$, donde $e$ es el elemento identidad
        de la multiplicación del grupo $G$ y $x^{i}$ es el producto de $i$
        veces $x$. Sabemos que $\mathbb{Z}^{*}_{10007} = \{\bar{1}, \bar{2}, 
        \bar{3}, ..., \overline{10006}\}$. En particular, el elemento identidad
        con respecto a la multiplicación en $\mathbb{Z}^{*}_{10007}$ es 
        $\bar{1}$ (se puede verificar con la tabla de multiplicación). Por la
        definición de órden, debemos hallar al más pequeño entero positivo $i$
        tal que $5^{i} \equiv 1 \pmod{10007}$.
        
        Dada la siguiente implementación que calcula al entero $i$ que estamos
        buscando y después lo compara con el órden del grupo 
        $\mathbb{Z}^{*}_{10007}$ para saber si son el mismo
        \begin{lstlisting}[language=Python]
        # Nos dice si el entero "a" es una raiz primitiva de Zp.
        def es_primitiva(a, p):
            for i in range (1, p):
                if(pow(a, i, p) == 1):
                    orden = i
                    break
            return (orden == p-1) and (mcd(a, p) == 1)
        
        def main():
            print(es_primitiva(5, 10007))
        
        if __name__ == "__main__":
            main()
        \end{lstlisting}
        
        obtenemos que $i = 10006$ (para este caso en particular el resultado
        se puede verificar con el \textit{Pequeño Teorema de Fermat}), y como 
        $ord(5) = 10006 = \varphi(10007)$ y $(5, 10007) = 1$ entonces la función
        \textit{es\_primitiva} regresa \textbf{True}. Por lo tanto, $5$ es una
        raíz primitiva de $\mathbb{Z}^{*}_{10007}$. 
        
        \newpage
        % Ejercicio 2.b
        \item $\rho_{1} = 4063$, $\rho_{2} = 5136$ y $\rho_{0} = 9865$. Úselos
        para calcular los logaritmos de $2$, $3$, y $7$ base $5$.

        ¿Por qué no es necesario calcular el logaritmo de $5$ base $\alpha$?
        
        \textsc{Solución:} Gracias a los valores de $\rho_{i}$ tenemos que 
        \begin{align*}
            5^{4063} \pmod{10007} &= 2 \times 3 \times 7 \\
            5^{5136} \pmod{10007} &= 2 \times 3^{3} \\
            5^{9865} \pmod{10007} &= 3^{3} \times 7
        \end{align*}
        
        de donde obtenemos
        \begin{align*}
            \log_{5} 2 + \log_{5} 3 + \log_{5} 7 &= 4063 \pmod{10006} \\
            \log_{5} 2 + 3 \log_{5} 3 &= 5136 \pmod{10006} \\
            3 \log_{5} 3 + \log_{5} 7 &= 9865\pmod{10006} 
        \end{align*}
        
        el cual es un sistema de ecuaciones con $3$ incógnitas ($\log_{5} 2$,
        $\log_{5} 3$ y $\log_{5} 7$).
        
        Resolvemos el sistema usando eliminación gaussiana:
        \begin{equation*}
            \begin{pmatrix}
            1 & 1 & 1 & | & 4063 \\
            1 & 3 & 0 & | & 5136 \\
            0 & 3 & 1 & | & 9865
            \end{pmatrix}
            \approx
            \begin{pmatrix}
            1 & 1 & 1 & | & 4063\\
            0 & 2 & 10005 & | & 1073 \\
            0 & 3 & 1 & | & 9865
            \end{pmatrix}
            \approx
            \begin{pmatrix}
            1 & 1 & 1 & | & 4063\\
            0 & 2 & 10005 & | & 1073\\
            0 & 1 & 6671 & | & 9959
            \end{pmatrix}
            \approx
        \end{equation*} 
        
        \begin{equation*}
            \begin{pmatrix}
            1 & 1 & 1 & | & 4063\\
            0 & 1 & 3334 & | & 1120\\
            0 & 1 & 6671 & | & 9959
            \end{pmatrix}
            \approx
            \begin{pmatrix}
            1 & 1 & 1 & | & 4063\\
            0 & 1 & 3334 & | & 1120\\
            0 & 0 & 3337 & | & 8839
            \end{pmatrix}
            \approx
            \begin{pmatrix}
            1 & 1 & 1 & | & 4063\\
            0 & 1 & 3334 & | & 1120\\
            0 & 0 & 1 & | & 1301
            \end{pmatrix}
            \approx
        \end{equation*}
        
        \begin{equation*}
            \begin{pmatrix}
            1 & 1 & 1 & | & 4063\\
            0 & 1 & 0 & | & 6190\\
            0 & 0 & 1 & | & 1301
            \end{pmatrix}
            \approx
            \begin{pmatrix}
            1 & 0 & 1 & | & 7879\\
            0 & 1 & 0 & | & 6190\\
            0 & 0 & 1 & | & 1301
            \end{pmatrix}
            \approx
            \begin{pmatrix}
            1 & 0 & 0 & | & 6578\\
            0 & 1 & 0 & | & 6190\\
            0 & 0 & 1 & | & 1301
            \end{pmatrix}
        \end{equation*}
        
        donde las operaciones realizadas fueron
        \begin{itemize}
            \item $F_{2} - F_{1} \rightarrow F_{2}$
            \item $6671 \cdot F_{3} \rightarrow F_{3}$ ($6671$ es el inverso
            multiplicativo de $3$ en $\pmod{10006}$).
            \item $F_{2} - F_{3} \rightarrow F_{2}$
            \item $F_{3} - F_{2} \rightarrow F_{3}$
            \item $4003 \cdot F_{3} \rightarrow F_{3}$ ($4003$ es el inverso
            multiplicativo de $3337$ en $\pmod{10006}$).
            \item $F_{2} - 3334 \cdot F_{3} \rightarrow F_{2}$
            \item $F_{1} - F_{2} \rightarrow F_{1}$
            \item $F_{1} - F_{3} \rightarrow F_{1}$
        \end{itemize}
        
        Por lo anterior, obtenemos que $\log_{5} 2 = 6578$, $\log_{5} 3 = 6190$ 
        y $\log_{5} 7 = 1301$.
     
        No es necesario calcular el logaritmo de $5$ base $\alpha = 5$ ya que 
        por definición el resultado es igual a $1$ ($x = log_{5} \; 5 
        \Leftrightarrow 5^{x} = 5 \Rightarrow x = 1$).

        % Ejercicio 2.c
        \item Dados los cálculos anteriores, obtener el índice de $\beta = 9451$
        en $\alpha$ módulo $10007$.
        
        \textsc{Solución:} Tomamos $k = 7736$ de manera \textit{aleatoria}.
        Calculamos
        \begin{equation*}
            \beta \alpha^{k} = 9451 \times 5^{7736} \pmod{10007} = 8400
            = 2^{4} \times 3 \times 5^{2} \times 7  
        \end{equation*}
        
        Como la expresión anterior se factoriza sobre $S = \{2, 3, 5, 7\}$,
        obtenemos 
        \begin{align*}
            \log_{5} 9451 
            &= 4 \log_{5} 2 + \log_{5} 3 + 2 \log_{5} 5 + \log_{5} 7 - s
               \pmod{10006}\\
            &= 4 (6578) + 6190 + 2 (1) + 1301 - 7736 \pmod{10006}
            && \text{por el inciso anterior} \\ 
            &= 6300 + 6190 + 2 + 1301 - 7736 \pmod{10006} \\
            &= 6057
        \end{align*}
        
        Así, el índice que buscamos es $6057$ (para verificarlo podemos 
        comprobar que efectivamente $5^{6057} \equiv 9451 \pmod{10007}$). 
    \end{enumerate}

    % Ejercicio 3.
    \item Descifrar el siguiente mensaje en $RSA$ con parámetros $(2257, 7)$ 
    con las siguientes condiciones:
    \begin{enumerate}
        % Ejercicio 3.a
        \item Aplicar el algoritmo de Criba Cuadrática para descomponer a $2257$.
        \begin{itemize}
            \item Usar la base $S = \{-1, 2, 3, 17\}$.
            \item Dar $M$, $B$ y decir para qué sirven.
            \item Explicar el proceso por el cual se obtienen $x$ e $y$, con los
            cuales se puede descomponer $2257$.
            \item Dar los valores para los cuales se obtienen $x$ e $y$ tales que
            $(x - y, 2257)$ es un factor no trivial de $2257$.
        \end{itemize}
        
        \textsc{Solución:} Primero calculamos los parámetros $M$ y $B$ que 
        determinarán el tamaño del intervalo de la criba y de la base de los 
        factores, respectivamente.
        \begin{equation*}
            M 
            = \floor{(e^{\sqrt{\ln(2257) \; \ln(\ln(2257))}})^
              {\frac{3 \sqrt{2}}{4}}}
            = 67
        \end{equation*}
        
        \begin{equation*}
            B 
            = \floor{(e^{\sqrt{\ln(2257) \; \ln(\ln(2257))}})^
              {\frac{\sqrt{2}}{4}}}
            = 4
        \end{equation*}
        
        De este modo ya tenemos el intervalo de la criba $[-67,67]$. Además, 
        podemos verificar que efectivamente la cardinalidad de la base de
        factores $S = \{-1, 2, 3, 17\}$ es igual a $4$.
        
        Luego, se buscan parejas $(a_{i}, b_{i})$ con $a^{2}_{i} \equiv 
        b_{i} \pmod{n}$. Para obtener estas parejas, haremos lo siguiente:
        
        Calculamos
        \begin{equation*}
            m = \floor{\sqrt{n}} = \floor{\sqrt{2257}} = 47
        \end{equation*}
        
        y formamos el polinomio
        \begin{equation*}
            q(x) = (x + m)^{2} - n = (x + 47)^{2} - 2257
        \end{equation*}
        
        Tenemos que $a_{i} = x + 47$ ya que 
        \begin{align*}
            a^{2} \equiv b_{i} 
            &= q(i) \\ 
            &= (i + 47) - 2257, \; \; i \in \{0, 1, -1, 2, -2, 3, -3, ...\}
        \end{align*}

        Debemos expresar $b_{i} = \prod^{s}_{k = 1} p^{e_{ij}}_{k}$. Esto lo
        podemos observar en la siguiente tabla:
        
        \begin{center}
            \begin{tabular}{|c|c|c|c|c|c|} 
            \hline
            $x$ & $b_{i}$ = $q(x)$ & $b_{i}$ = $\prod_{i=1}^{4} p_{i}$ & 
            $a_{i}$ & $w_{i}$ & $v_{i}$ \\ \hline
            $0$ & $-48$ & $(-1^1)(2^4)(3^1)(17^0)$ & $47$ & $(1,4,1,0)$ & 
            $(1,0,1,0)$ \\ \hline
            $1$ & & No se puede expresar en términos de la base & & & \\ \hline
            $-1$ & & No se puede expresar en términos de la base & & & \\ \hline
            $2$ & $144$ & $(-1^0)(2^4)(3^2)(17^0)$ & $49$ & $(0,4,2,0)$ & 
            $(0,0,0,0)$ \\ \hline
            $-2$ & & No se puede expresar en términos de la base & & & \\ \hline
            $3$ & $243$ & $(-1^0)(2^0)(3^5)(17^0)$ & $50$ & $(0,0,5,0)$ & 
            $(0,0,1,0)$ \\ \hline
            $-3$ & & No se puede expresar en términos de la base & & & \\ \hline
            $4$ & & No se puede expresar en términos de la base & & & \\ \hline
            $-4$ & $-408$ & $(-1^1)(2^3)(3^1)(17^1)$ & $43$ & $(1,3,1,1)$ & 
            $(1,1,1,1)$ \\ \hline
            $5$ & & No se puede expresar en términos de la base & & & \\ \hline
            $-5$ & & No se puede expresar en términos de la base & & & \\ \hline
            $6$ & & No se puede expresar en términos de la base & & & \\ \hline
            $-6$ & $-576$ & $(-1^1)(2^6)(3^2)(17^0)$ & $41$ & $(1,6,2,0)$ & 
            $(1,0,0,0)$ \\ \hline
            \end{tabular}
        \end{center}
        
        La tabla para fines prácticos la podemos ver de la siguiente forma:
         \begin{center}
            \begin{tabular}{|c|c|c|c|c|c|c|} 
            \hline
            $i$ & $x$ & $b_{i}$ = $q(x)$ & $b_{i}$ = $\prod_{i=1}^{4} p_{i}$ & 
            $a_{i}$ & $w_{i}$ & $v_{i}$ \\ \hline
            $1$ & $0$ & $-48$ & $(-1^1)(2^4)(3^1)(17^0)$ & $47$ & $(1,4,1,0)$ &
            $(1,0,1,0)$ \\ \hline
            $2$ & $2$ & $144$ & $(-1^0)(2^4)(3^2)(17^0)$ & $49$ & $(0,4,2,0)$ & 
            $(0,0,0,0)$ \\ \hline
            $3$ & $3$ & $243$ & $(-1^0)(2^0)(3^5)(17^0)$ & $50$ & $(0,0,5,0)$ & 
            $(0,0,1,0)$ \\ \hline
            $4$ & $-4$ & $-408$ & $(-1^1)(2^3)(3^1)(17^1)$ & $43$ & $(1,3,1,1)$ & 
            $(1,1,1,1)$ \\ \hline
            $5$ & $-6$ & $-576$ & $(-1^1)(2^6)(3^2)(17^0)$ & $41$ & $(1,6,2,0)$ & 
            $(1,0,0,0)$ \\ \hline
            \end{tabular}
        \end{center}
        
        En particular tenemos que $v_{1} + v_{3} + v_{5} = 0$. Por lo que, 
        $T = \{1, 3, 5\}$.
        
        Procedemos a calcular
        \begin{align*}
            x 
            &= a_{1} a_{3} a_{5} \pmod{2257} \\
            &= 47 \cdot 50 \cdot 41 \pmod{2257} \\
            &= 1556
        \end{align*}
        
        y los valores de $l_{j}$ (los cuales obtenemos al sumar coordenada a
        coordenada los vectores $w_{i}$ con $i \in T$, y el valor que obtenemos
        después de esta suma lo dividimos entre $2$)
        \begin{equation*}
            l_{1} = 1, l_{2} = 5, l_{3} = 4, l_{4} = 0
        \end{equation*}
        
        gracias a los cuales podemos calcular
        \begin{equation*}
            y = 1^{1} \cdot 2^{5} \cdot 3^{4} \cdot 7^{0} = 
            -2^{5} \cdot 3^{4} = 2592
        \end{equation*}
        
        Como $1556 \not \equiv \pm 2592 \pmod{2257}$ entonces sólo queda obtener
        \begin{equation*}
            (x - y, n) = (1556 - 2592, 2257)= (-1036, 2257) = 37
        \end{equation*}
        
        y
        \begin{equation*}
            (x + y, n) = (1556 + 2592, 2257)= (4148, 2257) = 61
        \end{equation*}
        
        Por lo tanto, $2257$ tiene dos factores no triviales que son $37$ y $61$. 
        
        % Ejercicio 3.b
        \newpage
        \item Descifrar el mensaje mediante el siguiente proceso:
        \begin{itemize}
            \item Dar las llaves pública y privada de $RSA$ y su proceso de cómo
            se obtienen de manera resumida pero clara.
            
            \textsc{Solución:} Sabemos que los parámetros del mensaje $RSA$ son
            $(2257, 7)$, por lo que $p = 37$ y $q = 61$ (por el inciso anterior).
            
            Entonces $n = pq = 37 \cdot 61 = 2257$, y obtenemos 
            \begin{align*}
                \varphi(n) 
                &= \varphi(37 \cdot 61) \\ 
                &= \varphi(37) \varphi(61) \\
                &= (37 - 1) (61 - 1)
                && \text{ya que $p$ y $q$ son primos} \\
                &= 36 \cdot 60 \\
                &= 2160
            \end{align*}
            
            Después, elegimos una llave de cifrado $e$, que es un entero impar
            (pues $(p-1)$ y $(q-1)$ son pares) tal que $0 < e < \varphi(n)$ y 
            es primo relativo con $\varphi(n)$. En este caso, por los 
            parámetros del mensaje $RSA$ tenemos que $e = 7$ (el cual, 
            efectivamente cumple las propiedades descritas anteriormente).
            
            Ahora, debemos calcular el número $d$ tal que $0 \leq d \leq n$ y 
            que cumple
            \begin{equation*}
                ed \equiv 1 \pmod{\varphi(n)}
            \end{equation*}
            
            Esto lo podemos obtener usando el algoritmo extendido de Euclides,
            ya que $d$ es el inverso multiplicativo de $e$ módulo $\varphi(n)$.
            
            Dada la siguiente implementación del algoritmo extendido de Euclides
            \begin{lstlisting}[language=Python]
            '''
            Regresa una tupla (mcd, s, t) que obtenemos al aplicar el 
            algoritmo extendido de Euclides, donde as + bt = mcd(a, b) 
            son los elementos que conforman la tupla.
            '''
            def aee(a, b):
                s = 0; s_i = 1
                t = 1; t_i = 0
                g = b; g_i = a
            
                while g != 0:
                    cociente = g_i // g
                    g_i, g = g, g_i - cociente * g
                    s_i, s = s, s_i - cociente * s
                    t_i, t = t, t_i - cociente * t
                return (g_i, s_i, t_i)
            
            # Calcula el inverso de 7 en Z_{2160}.
            def main():
                g, s, t = aee(7, 2160)
                # El inverso multiplicativo de a modulo m existe si y 
                # solo si (a,m) = 1.
                if g != 1:
                    print("No tiene inverso multiplicativo.")
                else:
                    inverso = s % 2160
                    print(inverso)
                
            if __name__ == "__main__":
                main()
            \end{lstlisting}
            
            obtenemos que $d = 1543$. 
            
            Por lo tanto, la llave pública es la tupla
            \begin{equation*}
                P = (e, n) = (7, 2160)
            \end{equation*}
            
            y la llave privada, que se mantiene secreta, es $d = 1543$, o bien,
            la tupla
            \begin{equation*}
                S = (d, n) = (1543, 2160)
            \end{equation*}
        
            \item Descifrar el mensaje. 
            \begin{verbatim}
            585 1660 585 2011 431 322 431 322 274 585 322 431 585 1660 
            68 322 1660 1933 1132 128 1995 322 2218 322 128 399 585 1660 
            128 399 322 585 2011 1933 1132 1411 2011 585 1660 128 399 233 
            233 322 2218 585 274 319 2011 585 1660 128 399 233 233 319 
            1660 319 2011 399 68 1660 399 1387 399 128 322 274 322 2218 
            399 2187 319 2011 399 68 1660 399 1387 399 128 322 585 1660 
            585 2011 431 585 128 322 2011 319 1418 1132 585 2011 322 233 
            585 128 319 1660 233 319 1 322 2011 399 128 319 1411 322 284 
            585 274 322 1660 1418 1132 585 585 2011 431 319 585 2011 431 
            322 1933 1132 1411 2187 399 284 585 274 431 399 2187 319
            \end{verbatim}
        
            \textsc{Solución:} Para recuperar el mensaje original a partir del 
            cifrado se realiza la siguiente operación:
            \begin{equation*}
                M = c^{d} \pmod{2257}
            \end{equation*}
            
            donde $c$ es el texto cifrado y $d$ es nuestra llave privada. 
            
            Dada la siguiente implementación
            \begin{lstlisting}[language=Python]
            # Descifra un mensaje utilizando RSA. 
            def descifrar(d, n):
                # Leemos el archvio de texto donde se encuentra el mensaje.
                file = open('texto.txt', 'r')
                data = file.readlines()
                file.close()
            
                msg = []
                # Calculamos c^d (mod n) para cada uno de los elementos 
                # del mensaje.
                for renglon in data:
                    for palabra in renglon.split(' '):
                        msg.append(pow(int(palabra), d, n))
                print(msg)
            
                modulo = []
                # Le aplicamos (mod 26) a cada uno de los elementos del 
                # mensaje descifrado.
                for elemento in msg:
                    modulo.append(elemento % 26)
                print(modulo)
                    
            def main():
                descifrar(1543, 2257)
            
            if __name__ == "__main__":
                main()
            \end{lstlisting}
            
            tenemos que la función \textit{descifrar} nos regresa dos listas:
            la primera es el resultado de aplicar la operación $c^{d} \pmod{n}$
            a cada uno de los elementos del mensaje, mientras que la segunda
            utiliza esta primer lista y le aplica módulo $26$ (ya que los
            elementos obtenidos al descifrar el mensaje están en un rango de 
            $[1,26]$ y al momento de sustituir queremos que los elementos en un
            rango de $[0,25]$). 
            
            La primer lista se ve como
            \begin{lstlisting}[language=Python]
            [4, 13, 4, 18, 19, 26, 19, 26, 17, 4, 26, 19, 4, 13, 
            6, 26, 13, 12, 20, 2, 7, 26, 15, 26, 2, 8, 4, 13, 2, 
            8, 26, 4, 18, 12, 20, 24, 18, 4, 13, 2, 8, 11, 11, 
            26, 15, 4, 17, 14, 18, 4, 13, 2, 8, 11, 11, 14, 13, 
            14, 18, 8, 6, 13, 8, 5, 8, 2, 26, 17, 26, 15, 8, 3, 
            14, 18, 8, 6, 13, 8, 5, 8, 2, 26, 4, 13, 4, 18, 19, 
            4, 2, 26, 18, 14, 16, 20, 4, 18, 26, 11, 4, 2, 14, 
            13, 11, 14, 1, 26, 18, 8, 2, 14, 24, 26, 21, 4, 17, 
            26, 13, 16, 20, 4, 4, 18, 19, 14, 4, 18, 19, 26, 12, 
            20, 24, 3, 8, 21, 4, 17, 19, 8, 3, 14]
            \end{lstlisting}
            
            mientras que la segunda lista es de la forma
            \begin{lstlisting}[language=Python]
            [4, 13, 4, 18, 19, 0, 19, 0, 17, 4, 0, 19, 4, 13, 6, 
            0, 13, 12, 20, 2, 7, 0, 15, 0, 2, 8, 4, 13, 2, 8, 0, 
            4, 18, 12, 20, 24, 18, 4, 13, 2, 8, 11, 11, 0, 15, 4, 
            17, 14, 18, 4, 13, 2, 8, 11, 11, 14, 13, 14, 18, 8, 
            6, 13, 8, 5, 8, 2, 0, 17, 0, 15, 8, 3, 14, 18, 8, 6, 
            13, 8, 5, 8, 2, 0, 4, 13, 4, 18, 19, 4, 2, 0, 18, 14, 
            16, 20, 4, 18, 0, 11, 4, 2, 14, 13, 11, 14, 1, 0, 18, 
            8, 2, 14, 24, 0, 21, 4, 17, 0, 13, 16, 20, 4, 4, 18, 
            19, 14, 4, 18, 19, 0, 12, 20, 24, 3, 8, 21, 4, 17, 19, 
            8, 3, 14]
            \end{lstlisting}
            
            Finalmente, usaremos esta segunda lista para obtener el mensaje 
            original.
            
            Dada la siguiente tabla, en donde asignamos a cada letra
            un número 
            \begin{center}
            \begin{tabular}{|c|c|c|c|c|c|c|c|c|c|c|c|c|c|}
            \hline
            A & B & C & D & E & F & G & H & I & J & K  & L  & M  & N 
            \\ \hline
            0 & 1 & 2 & 3 & 4 & 5 & 6 & 7 & 8 & 9 & 10 & 11 & 12 & 13
            \\ \hline
            \end{tabular}
            \end{center}
            \begin{center}
            \begin{tabular}{|c|c|c|c|c|c|c|c|c|c|c|c|c|c|}
            \hline
            O  & P  & Q  & R  & S  & T  & U  & V  & W  & X  & Y  & Z  
            \\ \hline
            14 & 15 & 16 & 17 & 18 & 19 & 20 & 21 & 22 & 23 & 24 & 25
            \\ \hline
            \end{tabular}
            \end{center}
            
            podemos concluir que el mensaje es:
            \begin{verbatim}
            EN ESTA TAREA TENGAN MUCHA PACIENCIA ES MUY SENCILLA PERO 
            SENCILLO NO SIGNIFICA RAPIDO SIGNIFICA EN ESTE CASO QUE 
            SALE CON LO BASICO YA VERAN QUE ESTO ESTA MUY DIVERTIDO.
            \end{verbatim}
    \end{itemize}
    \end{enumerate}
\end{enumerate}
\end{document}