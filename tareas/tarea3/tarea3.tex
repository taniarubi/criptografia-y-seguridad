\documentclass[letterpaper,11pt]{article}

% Soporte para los acentos.
\usepackage[utf8]{inputenc}
\usepackage[T1]{fontenc}
% Idioma español.
\usepackage[spanish,mexico, es-tabla]{babel}
% Soporte de símbolos adicionales (matemáticas)
\usepackage{multirow}
\usepackage{amsmath}
\usepackage{amssymb}
\usepackage{amsthm}
\usepackage{amsfonts}
\usepackage{mathtools}
\DeclarePairedDelimiter\floor{\lfloor}{\rfloor}
\usepackage{latexsym}
\usepackage{enumerate}
\usepackage{ragged2e}
\usepackage{listings}
\usepackage{xcolor}
\usepackage{array}
% Modificamos los márgenes del documento.                                       %
\usepackage[lmargin=2cm,rmargin=2cm,top=2cm,bottom=2cm]{geometry}

\definecolor{codegreen}{rgb}{0,0.6,0}
\definecolor{codegray}{rgb}{0.5,0.5,0.5}
\definecolor{codepurple}{rgb}{0.58,0,0.82}
\definecolor{backcolour}{rgb}{0.95,0.95,0.92}

\lstdefinestyle{mystyle}{
    backgroundcolor=\color{backcolour},   
    commentstyle=\color{codegreen},
    keywordstyle=\color{magenta},
    numberstyle=\tiny\color{codegray},
    stringstyle=\color{codepurple},
    basicstyle=\ttfamily\footnotesize,
    breakatwhitespace=false,         
    breaklines=true,                 
    captionpos=b,                    
    keepspaces=true,                 
    numbers=left,                    
    numbersep=5pt,                  
    showspaces=false,                
    showstringspaces=false,
    showtabs=false,                  
    tabsize=2
}

\lstset{style=mystyle}

\title{Facultad de Ciencias, UNAM \\ Criptografía y Seguridad \\ Tarea 3}
\author{Altamirano Vázquez Jesús Fernando \\
        Rubí Rojas Tania Michelle}
\date{\today}

\begin{document}
\maketitle

\begin{enumerate}
    % Ejercicio 1.
    \item Sea $\mathbb{E} : y^2 + 20x = x^3 + 21 \pmod{35}$ y sea $Q = (15, -4) 
    \in \mathbb{E}$.
    \begin{enumerate}
        % Ejercicio 1.a
        \item Factoriza $35$ tratando de calcular $3Q$.

        \textsc{Solución:} Recordemos la definición de suma de puntos:
        
        \begin{equation*}
            P + Q =
            \begin{cases}
                \infty & \text{Si $x_1 = x_2$ y $-y_1 = y_2$}\\
                (x_3,y_3) & \text{$x_3 = \lambda^2 - x_1 - x_2$ y $y_3 = 
                \lambda (x_1 - x_3) - y_1$}
            \end{cases}       
        \end{equation*}
        
        Al igual que debemos recordar como calcular $\lambda$ :
        \begin{equation*}
            \lambda =
            \begin{cases}
                3x_1^2 + A \cdot (2y_1)^{-1} \pmod{p} & 
                \text{si $P = Q$} \\
                (y_1-y_2) \cdot (x_1 - x_2)^{-1} \pmod{p} & 
                \text{en otro caso}
            \end{cases}       
        \end{equation*}

        Además, tenemos que $\mathbb{E} : y^2 = x^3 - 20x + 21$. Primero, 
        obtendremos $2Q$, para lo cual obtendremos primero el valor de 
        $\lambda$. Como $Q = Q$, entonces 
        \begin{align*}
            \lambda
            &= 3(15)^2 + (-20) \cdot 2(-4)^{-1} \pmod{35} \\ 
            &= (3 \cdot 255) - 20 \cdot (-8)^{-1} \pmod{35} \\
            &= 655 \cdot 13 \pmod{35} \\ 
            &= 8515 \pmod{35} \\
            &= 10
        \end{align*}
        
        Ahora, calculamos el valor de $x_3$:
        \begin{align*}
            x_3
            &= 10^2 - 15 - 15 \pmod{35} \\ 
            &= 100 - 30 \pmod{35} \\
            &= 70 \pmod{35} \\ 
            &= 0
        \end{align*}
        
        Finalmente, calculamos $y_3$:
        \begin{align*}
            x_3
            &= 10(15 - 0) - (-4) \pmod{35} \\ 
            &= 150 + 4 \pmod{35} \\
            &= 154 \pmod{35} \\ 
            &= 14
        \end{align*}
        
        Por lo tanto, $2Q = (0,14)$. Ahora bien, procedemos a calcular 
        $3Q = 2Q + Q$, y para ello realizamos los mismos pasos que en el 
        cálculo anterior: como $2Q \neq Q$ entonces 
        \begin{align*}
            \lambda 
            &= (14 - (-4)) \cdot (0 - 15)^{-1} \pmod{35} \\ 
            &= 18 \cdot (-15)^{-1} \pmod{35}
        \end{align*}

        Pero, $15$ no tiene inverso multiplicativo en este caso, por lo que 
        debemos encontrar el máximo común divisor de $(35, 15)$, el cual es 
        $5$. De esta forma, determinamos que $5$ es un factor de $35$.
    
        % Ejercicio 1.b
        \item Factoriza $35$ tratando de calcular $4Q$ duplicándolo.

        \textsc{Solución:} Como ya tenemos calculado $2Q$, entonces haremos 
        $2Q + 2Q$ para obtener $4Q$. Así, 
        \begin{align*}
             \lambda 
             &= 3(0)^2 + (-20) \cdot (2 \cdot 14)^{-1} \pmod{35} \\
             &= -20 \cdot (28)^{-1}
        \end{align*}
        
        Pero, $28 = 2 \cdot 14$ no tiene inverso multiplicativo, por lo que 
        debemos calcular el $MCD(28,35)$, el cual es $7$. De esta forma, podemos 
        concluir que $7$ es un factor de $35$.

        % Ejercicio 1.c
        \item Calcula $3Q$ y $4Q$ sobre $\mathbb{E} \pmod{5}$ y sobre 
        $\mathbb{E} \pmod{7}$. Explica por qué el factor $5$ se obtiene 
        calculando $3Q$ y por qué el factor $7$ se obtiene calculando $4Q$.
        
        \textsc{Solución:}
        \begin{itemize}
            \item $3Q$ sobre $\mathbb{E} \pmod{5}$. Tenemos que  
            \begin{align*}
                \mathbb{E}: y^2 
                &= x^3 - 20x + 21 \pmod{35} \\ 
                &= x^3 + 1 \pmod{35}
            \end{align*}

            Ahora bien, primero calcularemos $2Q$. Como $Q = (15, -4) = (0, 1) 
            = Q$, entonces
            \begin{align*}
                \lambda 
                &= (3(0)^2 + 0) \cdot (2 \cdot 1)^{-1} \pmod{5} \\ 
                &= 0 \cdot (2)^{-1} \pmod{5} \\
                &= 0 \cdot 3 \pmod{5} \\ 
                &= 0
            \end{align*}

            Luego, calculamos $x_3$:
            \begin{equation*}
                x_3 = (0)^2 - 0 - 0 \pmod{5} = 0
            \end{equation*}

            Finalmente, calculamos $y_3$:
            \begin{align*}
                y_3
                &= 0(0 - 0) - 1 \pmod{5} \\ 
                &= 0 - 1 \pmod{5} \\ 
                &= 4
            \end{align*}

            Por lo tanto, $2Q = (0, 4)$.

            Para calcular $3Q$ haremos $2Q + Q$. Como los puntos son diferentes,
            entonces
            \begin{align*}
                \lambda 
                &= (4 - 1) \cdot (0 - 0)^{-1} \pmod{5} \\ 
                &= 5 \cdot 0 \pmod{5} \\ 
                &= 0
            \end{align*}

            Como $MCD(5, 0) = 5$, entonces tenemos que $5$ es un factor de $5$.

            \item $4Q$ sobre $\mathbb{E} \pmod{7}$. Tenemos que 
            \begin{align*}
                \mathbb{E} : y^2 
                &= x^3 - 20x + 21 \pmod{7} \\ 
                &= x^3 + x \pmod{7}
            \end{align*}

            Ahora bien, primero calcularemos $2Q$. Como $Q = (15, -4) = (1, 3)
            = Q$ entonces 
            \begin{align*}
                \lambda
                &= (3(1)^2 + 1) \cdot (2 \cdot 3)^{-1} \pmod{7} \\
                &= (3 + 1) \cdot (6)^-1 \pmod{7} \\
                &= 4 \cdot 6 \pmod{7} \\
                &= 24 \pmod{7} \\ 
                &= 3
            \end{align*}

            Calculamos $x_3$:
            \begin{align*}
                x_3
                &= (3)^2 - 1 - 1 \pmod{7} \\ 
                &= 9 - 2 \pmod{7} \\ 
                &= 7 \pmod{7} \\
                &= 0
            \end{align*}

            Calculamos $y_3$:
            \begin{align*}
                y_3
                &= 3(1 - 0) - 3 \pmod{7} \\ 
                &= 3 - 3 \pmod{7} \\ 
                &= 0
            \end{align*}

            Por lo tanto, $2Q = (0, 0)$.

            Ahora bien, para calcular $4Q$ haremos $2Q + 2Q$. Como los dos puntos 
            son iguales, entonces 
            \begin{align*}
                \lambda 
                &= (3(0)^2 + 1) \cdot (2 \cdot 0)^{-1} \pmod{7} \\ 
                &= 1 \cdot (0)^{-1} \pmod{7}
            \end{align*}

            Pero $0$ no tiene inverso multiplicativo en este caso, y como 
            $MCD(7, 0) = 7$ entonces tenemos que $7$ es un factor de $7$.
        \end{itemize}
    \end{enumerate}

    En cada suma se verifica si el máximo común divisor de ($5$ o $7$) y $k$ es 
    diferente de $1$ (donde $k$ es el número al cual le sacaremos el inverso en 
    la lambda). Si el $MCD$ no es $1$, eso significa que existe un entero $n$ 
    que divide a $5$ y a $k$, pero como $5$ y $7$ son primos, eso quiere decir 
    que ellos mismos son los números que se dividen. Por este motivo, obtenemos 
    a $5$ y $7$ como factores.

    % Ejercicio 2.
    \item Sea $\mathbb{E}$ la curva elíptica $y^2 = x^3 + x + 28$ definida sobre 
    $\mathbb{Z}_{71}$. 
    \begin{enumerate}
        % Ejercicio 2.a
        \item Calcula y muestra el número de puntos de $\mathbb{E}$.
        
        \textsc{Solución:} Sabemos que los puntos en una curva elíptica 
        \begin{equation*}
            E : y^2 = x^3 + Ax + B \pmod{n}
        \end{equation*}
        
        son los pares $(x, y) \pmod{n}$ con $x, y \in K$ tales que satisfacen
        la ecuación anterior, junto con el punto en el infinito. 
        
        \newpage
        Dada la siguiente implementación
        \begin{lstlisting}[language = Python]
        '''
        Regresa los puntos que pertenecen a la curva eliptica 
        E: y^2 = x^3 + Ax + B (mod n)
        '''
        def encontrar_puntos(A, B, n):
            puntos = []
            # Sabemos que x pertenece al conjunto [0, 70].
            for i in range (0, n):
                # Sabemos que y pertenece al conjunto [0, 70]. 
                for j in range (0, n):
                    # Encontramos el valor de x^3 + Ax + B (mod n)
                    valor = (pow(i, 3, n) + ((A * i) % n) + B) % n
                    # Encontramos los posibles valores para y.
                    y2 = pow(j, 2)
                    # Verificamos que el par satisface la ecuacion.
                    if (((y2 - valor) % n) == 0):
                        puntos.append((i, j))
            
            print("La curva eliptica E tiene " + str(len(puntos) + 1) + 
                  " puntos.")
            return puntos
            
        if __name__ == "__main__":
            print(encontrar_puntos(1, 28, 71))
        \end{lstlisting}
        
        obtenemos que, junto con el punto en el infinito, hay $72$ puntos
        en la curva elíptica, los cuales son:
        
        \begin{center}
        \begin{tabular}{c|c|c|c|c|c|c|c|c}
        (1, 32) & (1, 39) & (2, 31) & (2, 40) & (3, 22) & (3, 49) & (4, 5) & 
        (4, 66) & (5, 4) \\ \hline
        (5, 67) & (6, 26) & (6, 45) & (12, 8) & (12, 63) & (13, 26) & 
        (13, 45) & (15, 9) & (15, 62) \\ \hline
        (19, 27) & (19, 44) & (20, 5) & (20, 66) & (21, 3) & (21, 68) & 
        (22, 30) & (22, 41) & (23, 19) \\ \hline
        (23, 52) & (25, 22) & (25, 49) & (27, 0) & (31, 32) & (31, 39) & 
        (33, 1) & (33, 70) & (34, 23) \\ \hline
        (34, 48) & (35, 14) & (35, 57) & (36, 12) & (36, 59) & (37, 33) & 
        (37, 38) & (39, 32) & (39, 39) \\ \hline
        (41, 7) & (41, 64) & (43, 22) & (43, 49) & (47, 5) & (47, 66) & 
        (48, 11) & (48, 60) & (49, 24) \\ \hline
        (49, 47) & (52, 26) & (52, 45) & (53, 0) & (58, 27) & (58, 44) & 
        (61, 15) & (61, 56) & (62, 0) \\ \hline
        (63, 17) & (63, 54) & (65, 27) & (65, 44) & (66, 18) & (66, 53) & 
        (69, 35) & (69, 36) & $\infty$ \\
        \end{tabular}
        \end{center}

        % Ejercicio 2.b
        \item Muestra que $\mathbb{E}$ no es un grupo cíclico.

        \begin{proof}
            Sabemos que un elemento $P \in E$ es un punto primitivo si genera 
            a todo el conjunto de puntos que pertenecen a $E$, es decir, todos
            los elementos del grupo pueden ser expresados de la forma
            \begin{equation*}
                P + P + \cdots + P (\text{$k$ veces}) \; \; \; \; \; 
                \text{para alguna $k \in \{1, 2, ..., \#E(\mathbb{F}_q)\}$}
            \end{equation*}
            
            Por un corolario del teorema de \textit{Lagrange} sabemos que el órden
            de un subgrupo generado por un elemento en $E$ necesariamente divide 
            a $\#E(\mathbb{F}_q) = 72$, por lo que 
            \begin{equation*}
                D = \{1, 2, 3, 4, 6, 8, 9, 12, 18, 24, 36\}
            \end{equation*}
            
            son los posibles valores para el órden de cada uno de los puntos en 
            $E(\mathbb{F}_{71})$, ya que son justamente todos los divisores de 
            $72$. 
            
            Por el ejercicio \textit{2.c} sabemos que el órden de cada uno de los
            elementos en $E$ se encuentra dentro del conjunto
            \begin{equation*}
                O = \{1, 2, 3, 4, 6, 9, 12, 18, 36\}
            \end{equation*}
            
            Entonces, como 
            \begin{equation*}
                \#E(\mathbb{F}_{71}) = n = 72 = 2^3 \cdot 3^2
            \end{equation*}
            
            no es primo, podemos buscar al punto primitivo $P$ de la siguiente
            forma: para cada primo $p$ que divide a 72, calculamos
            $(\frac{n}{p})P$. Si ninguno de estos puntos es el punto en el
            infinito, entonces $P$ genera a $E$. Por ejemplo, si $P = (1, 32)$
            tenemos que 
            \begin{align*}
                \left(\frac{72}{2}\right) P &= 36(1, 32) = \infty \\
                \left(\frac{72}{3}\right) P &= 24(1, 32) = (20, 5)
            \end{align*}
            
            Como $36P = \infty$, entonces no genera a $E$.
            
            Ahora bien, podemos seguir el siguiente camino: el órden de un
            elemento en $E$ siempre será múltiplo de $36$ o $24$, y eso implica 
            que $kP$, con $k = |P| \cdot i = 36 \; o \; 24$, genera el punto en el
            infinito. Tomando un elemento $P$ cuyo órden pertenezca al conjunto
            $O$, tenemos que 
            
            \begin{equation*}
                |P = (1, 32)| = 18 \Rightarrow 18 \cdot 2 = 36 \Rightarrow 
                36P = \infty 
            \end{equation*}
            \begin{equation*}
                |P = (2, 31)| = 6 \Rightarrow 6 \cdot 4 = 24 \Rightarrow 
                24P = \infty
            \end{equation*}
            \begin{equation*}
                |P = (3, 22)| = 12 \Rightarrow 12 \cdot 2 = 24 \Rightarrow
                24P = \infty
            \end{equation*}
            \begin{equation*}
                |P = (4, 5)| = 36 \Rightarrow 36 \cdot 1 = 36 \Rightarrow
                36P = \infty
            \end{equation*}
            \begin{equation*}
                |P = (5, 4)| = 4 \Rightarrow 4 \cdot 6 = 24 \Rightarrow
                24P = \infty
            \end{equation*}
            \begin{equation*}
                |P = (20, 5)| = 3 \Rightarrow 3 \cdot 8 = 24 \Rightarrow
                24P = \infty
            \end{equation*}
            \begin{equation*}
                |P = (27, 0)| = 2 \Rightarrow 2 \cdot 12 = 24 \Rightarrow
                24P = \infty
            \end{equation*}
            \begin{equation*}
                |P = (31, 32) = 9 \Rightarrow 9 \cdot 4 = 36 \Rightarrow
                36P = \infty
            \end{equation*}
            
            Notemos que todas las operaciones fueron verificadas con la función 
            \texttt{suma\_puntos()}, implementada en el ejercicio \textit{2.c}. 
            
            Como $|\infty| = 1$ no puede ser el generador del grupo, y el órden
            del resto de los puntos en $E$ oscila entre los valores del conjunto
            $O$, eso quiere decir que todos los puntos generan el punto en el 
            infinito, lo que implica que $E$ no es generado por ninguno de sus 
            elementos. Por lo tanto, $E$ no es cíclico.
            
        \end{proof}

        % Ejercicio 2.c
        \item ¿Cuál es el máximo órden de un elemento en $\mathbb{E}$? Encuentra 
        un elemento que tenga este órden. 

        \textsc{Solución:} Por un corolario del teorema de \textit{Lagrange}
        sabemos que el órden de un punto siempre divide el órden del grupo
        $E(\mathbb{F}_{71})$. 
        
        Dada la siguiente implementación
        \begin{lstlisting}[language=Python]
        # Regresa los divisores de un numero n.
        def divisores(n):
            div = []
            for i in range (1, n):
                if ((n % i) == 0):
                    div.append(i)
            
            return div
        
        if __name__ == "__main__":
            print(divisores(72))
        \end{lstlisting}
        
        sabemos que 
        \begin{verbatim}
            D = [1, 2, 3, 4, 6, 8, 9, 12, 18, 24, 36]
        \end{verbatim}
        
        son los posibles valores para el órden de cada uno de los puntos 
        en $E (\mathbb{F}_{71})$. En particular, por el teorema de 
        \textit{Lagrange} tenemos que $\#E (\mathbb{F}_{71})P = 72P = \infty$,
        con $P \in E (\mathbb{F}_{71})$. 
        
        Ahora bien, el órden de un punto en $E (\mathbb{F}_{71})$ será el 
        mínimo entero $k \in D$ tal que $kP = \infty$.  
        
        Dada la siguiente implementación
        \begin{lstlisting}[language=Python]
        '''
        Regresa una tupla (mcd, s, t) que obtenemos al aplicar el algoritmo 
        extendido de Euclides, donde as + bt = mcd(a, b) son los elementos 
        que conforman la tupla.
        '''
        def aee(a, b):
            s = 0; s_i = 1
            t = 1; t_i = 0
            g = b; g_i = a
        
            while g != 0:
                cociente = g_i // g
                g_i, g = g, g_i - cociente * g
                s_i, s = s, s_i - cociente * s
                t_i, t = t, t_i - cociente * t
            return (g_i, s_i, t_i)
            
        # Regresa el inverso multiplicativo de a modulo m.
        def inverso(a, m):
            g, s, t = aee(a, m)
            # El inverso de a modulo m existe si y solo si (a, m) = 1.
            if g != 1:
                print("No tiene inverso multiplicativo.")
            else:
                inverso = s % m
            
            return inverso
            
        # Regresa la suma de dos puntos P y Q en E. 
        def suma_puntos(P, Q, A, p):
            # Casos especiales.
            if (P == None):
                return Q
            if (Q == None):
                return  P
                
            x1, y1 = P
            x2, y2 = Q
            
            if (x1 == x2):
                m = (3 * x1 * x1 + A) * inverso(2 * y1, p)
            else:
                m = (y1 - y2) * inverso(x1 - x2, p)
                
            x3 = m * m - x1 - x2
            y3 = m * (x1 - x3) - y1
            suma = (x3 % p, y3 % p)
            return suma
        


        # Regresa el orden de un elemento en E.
        def orden(P, a, p):
            # Como P = P y y_1 = 0 entonces P+P = infinito.
            if(P[1] == 0):
                return 2
            
            # Calculamos 2P.
            P2 = suma_puntos(P, P, a, p)
            aux = P2
            
            orden = 3
            for i in range(0, p):
                # Calculamos P + kP
                aux = suma_puntos(P, aux, a, p)
                # Si encontramos a 2P, entonces hemos encontrado el orden.
                if(aux == P2):
                    break
                orden += 1
            return orden
        
        # Regresa el orden de cada uno de los elementos en la lista puntos.
        def get_ordenes(puntos, a, p):
            ordenes = []
            for punto in puntos:
                o = orden(punto, a, p)
                ordenes.append(o)
            
            return ordenes
            
        if __name__ == "__main__":
            print(get_ordenes(encontrar_puntos(1, 28, 71), 1, 71))
        \end{lstlisting}
        
        obtenemos la siguiente tabla, la cual indica el órden de cada uno de
        los elementos que pertenecen a $E$.
        \begin{center}
        \begin{tabular}{c|c|c|c|c|c|c|c|c}
        $|(1, 32)| = 18$ & $|(1, 39)| = 18$ & $|(2, 31)| = 6$ & $|(2, 40)| = 6$ & 
        $|(3, 22)| = 12$ \\ \hline
        $|(3, 49)| = 12$ & $|(4, 5)| = 36$ & $|(4, 66)| = 36$ & $|(5, 4)| = 4$ &
        $|(5, 67)| = 4$  \\ \hline
        $|(6, 26)| = 18$ & $|(6, 45)| = 18$ & $|(12, 8)| = 18$ & 
        $|(12, 63)| = 18$ & $|(13, 26)| = 36$ \\ \hline
        $|(13, 45)| = 36$ & $|(15, 9)| = 36$ & $|(15, 62)| = 36$ & 
        $|(19, 27)| = 6$ & $|(19, 44)| = 6$ \\ \hline
        $|(20, 5)| = 3$ & $|(20, 66)| = 3$ & $|(21, 3)| = 36$ & 
        $|(21, 68)| = 36$ &  $|(22, 30)| = 18$ \\ \hline
        $|(22, 41)| = 18$ & $|(23, 19) = 36$ & $|(23, 52)| = 36$ & 
        $|(25, 22)| = 18$ & $|(25, 49)| = 18$ \\ \hline
        $|(27, 0)| = 2$ & $|(31, 32)| = 9$ & $|(31, 39)| = 9$ &  
        $|(33, 1)) = 36$ & $|(33, 70)| = 36$ \\  \hline
        $|(34, 23)| = 36$ & $|(34, 48)| = 36$ & $|(35, 14)| = 12$ & 
        $|(35, 57)| = 12$ & $|(36, 12)| = 9$ \\  \hline
        $|(36, 59)| = 9$ & $|(37, 33)| = 36$ & $|(37, 38)| = 36$ & 
        $|(39, 32)| = 6$ & $|(39, 39)| = 6$ \\ \hline
        $|(41, 7)| = 36$ & $|(41, 64)| = 36$ & $|(43, 22)| = 36$ & 
        $|(43, 49)| = 36$ & $|(47, 5)| = 36$ \\ \hline 
        $|(47, 66)| = 36$ & $|(48, 11)| = 36$ & $|(48, 60)| = 36$ & 
        $|(49, 24)| = 4$ & $|(49, 47)| = 4$ \\ \hline  
        $|(52, 26)| = 12$ & $|(52, 45)| = 12$ & $|(53, 0)| = 2$ & 
        $|(58, 27)| = 18$ & $|(58, 44)| = 18$ \\ \hline 
        $|(61, 15)| = 18$ & $|(61, 56)| = 18$ & $|(62, 0)| = 2$ & 
        $|(63, 17)| = 9$ & $|(63, 54)| = 9$ \\ \hline 
        $|(65, 27)| = 18$ & $|(65, 44)| = 18$ & $|(66, 18)| = 12$ & 
        $|(66, 53)| = 12$ & $|(69, 35)| = 18$ \\ \hline 
        $|(69, 36)| = 18$ & $|\infty| = 1$ 
        \end{tabular}
        \end{center}
        
        Por lo tanto, el punto $P = (4, 5)$ con $|P| = 36$ es un
        elemento en $E$ con el máximo órden.
    \end{enumerate}

    \newpage
    % Ejercicio 3.
    \item Sea $\mathbb{E} : y^2 - 2 = x^3 + 333x$ sobre $\mathbb{F}_{347}$ y 
    sea $P = (110, 136)$. 
    \begin{enumerate}
        % Ejercicio 3.a
        \item ¿Es $Q = (81, -176)$ un punto de $\mathbb{E}$?
        
        \textsc{Solución:} Para saber si $Q$ es un punto en $E$ simplemente 
        debemos comprobar que con los valores $x = 81$ y $y = -176$ se cumple
        la congruencia 
        \begin{equation*}
            y^2 - 2 \equiv x^3 + 333x \pmod{347}
        \end{equation*}
        
        Entonces
        \begin{align*}
            y^2 - 2 &\equiv x^3 + 333x \pmod{347} \\
            (-176)^2 - 2 &\equiv (81)^3 + 333(81) \pmod{347} \\
            30 974 &\equiv 558 414 \pmod{347} \\ 
        \end{align*}
        
        Como $p = 347 \; | \; 558414 - 30974 = 527 440$ ya que 
        $347(1520) = 527 440$, entonces podemos concluir que 
        $Q = (81, -176) \in E(F_{347})$.

        % Ejercicio 3.b
        \item Si sabemos que $|\mathbb{E}| = 358$. ¿Podemos decir que 
        $\mathbb{E}$ es criptográficamente útil? ¿Cuál es el órden de $P$? 
        ¿Entre qué valores se puede escoger la clave privada?

        \textsc{Solución:} Como $|E| = 358 = 2 \cdot 179$ entonces se puede
        decir que $E$ no es criptográficamente útil ya que una 
        \textit{curva fuerte} busca que 
        \begin{itemize}
            \item El órden de $E$ sea divisible por un primo grande ($2$ no lo 
            es), ó 
            \item El órden de $E$ sea un primo grande (tenemos que $358$ es par).
        \end{itemize}
      
        Utilizando la función \texttt{orden()}, definida en el ejercicio 
        \textit{2.c}, obtenemos que el órden del elemento $P = (110, 136)$ es
        $179$. 
        
        Los valores de la clave privada se pueden escoger entre los enteros
        del intervalo $[1, 179]$, el cual está acotado por el órden del punto 
        $P$ (ya que es el punto primitivo).

        % Ejercicio 3.c
        \item Si tu clave privada es $k = 101$ y algún conocido te ha enviado 
        el mensaje cifrado
        \begin{equation*}
            (M_1 = (232, 278), M_2 = (135, 214))
        \end{equation*}

        ¿Cuál era el mensaje original?

        \textsc{Solución:} El mensaje está cifrado usando \textit{ElGamal 
        elíptico}, por lo que simplemente hay que seguir su algoritmo para
        descifrar.
        \begin{itemize}
            \item Calculamos $M = M_2 - kM_1$, donde $k$ es nuestra llave 
            privada. 
            
            Usando la función \texttt{suma\_puntos()}, implementada en el
            ejercicio \textit{2.c}, obtenemos que 
            \begin{align*}
                M
                &= (135, 214) - 101(232, 278) \\ 
                &= (135, 214) - (275, 176) \\
                &= (135, 214) + (275, -176) 
                && \text{por definición de $-P$} \\ 
                &= (74, 87)
            \end{align*}
        \end{itemize}
        
        Por lo tanto, el mensaje original es $M = (74, 87)$.

    \end{enumerate}

    \newpage
    % Ejercicio 4.
    \item Sea $\mathbb{E} : F(x, y) = y^2 - x^3 - 2x - 7$ sobre 
    $\mathbb{Z}_{31}$ con $\# \mathbb{E} = 39$ y $P = (2, 9)$ es un punto de 
    órden $39$ sobre $\mathbb{E}$, el \textsc{ECIES} simplificado definido sobre 
    $\mathbb{E}$ tiene $\mathbb{Z}^{*}_{31}$ como espacio de texto plano, 
    supongamos que la clave privada es $m = 8$. 
    \begin{enumerate}
        % Ejercicio 4.a
        \item Calcula $Q = mP$.
        
        \textsc{Solución:} Usando la función \texttt{suma\_puntos()}, implementada
        en el ejercicio \textit{2.c}, obtenemos que 
        \begin{equation*}
            Q = 8(2, 9) = (8, 15)
        \end{equation*}

        % Ejercicio 4.b
        \item Descifra la siguiente cadena de texto cifrado 
        \begin{equation*}
            ((18, 1), 21), ((3, 1), 18), ((17, 0), 19), ((28, 0), 8)
        \end{equation*}

        \textsc{Solución:} Primero, desciframos la tupla $((18, 1), 21)$. 
        Como $18 \in \mathbb{Z}_{31}$ y $1 \in \mathbb{Z}_{2}$, entonces debemos 
        evaluar a $18$ en $y^2 = x^3 + 2x + 7 \pmod{31}$. Así,  
        \begin{align*}
            y^2 &= 18^3 + 2(18) + 7 \pmod{31} \\ 
            y^2 &= 5832 + 36 + 7 \pmod{31} \\ 
            y^2 &= 5875 \pmod{31} \\ 
            y^2 &\equiv 16
        \end{align*}

        Por lo que $y = \pm 4 \pmod{31}$. Notemos que el segundo elemento de 
        la tupla $(18, 1)$ nos dice que $y \equiv 1 \pmod{2}$, pero 
        $-4 \equiv 27 \pmod{31}$ pero $27 \not \equiv 1 \pmod{2}$.

        % Ejercicio 4.c
        \item Supongamos que cada texto plano representa un carácter alfabético,
        convierte el texto plano en una palabra en Inglés. Usa la asociación 
        $(A \rightarrow 1, ..., Z \rightarrow 26)$.
    \end{enumerate}
\end{enumerate}

\end{document}
