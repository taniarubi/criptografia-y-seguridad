\documentclass[letterpaper,11pt]{article}

% Soporte para los acentos.
\usepackage[utf8]{inputenc}
\usepackage[T1]{fontenc}
% Idioma español.
\usepackage[spanish,mexico, es-tabla]{babel}
% Soporte de símbolos adicionales (matemáticas)
\usepackage{multirow}
\usepackage{amsmath}
\usepackage{amssymb}
\usepackage{amsthm}
\usepackage{amsfonts}
\usepackage{mathtools}
\DeclarePairedDelimiter\floor{\lfloor}{\rfloor}
\usepackage{latexsym}
\usepackage{enumerate}
\usepackage{ragged2e}
\usepackage{listings}
\usepackage{xcolor}
\usepackage{array}
% Modificamos los márgenes del documento.                                       %
\usepackage[lmargin=2cm,rmargin=2cm,top=2cm,bottom=2cm]{geometry}

\definecolor{codegreen}{rgb}{0,0.6,0}
\definecolor{codegray}{rgb}{0.5,0.5,0.5}
\definecolor{codepurple}{rgb}{0.58,0,0.82}
\definecolor{backcolour}{rgb}{0.95,0.95,0.92}

\lstdefinestyle{mystyle}{
    backgroundcolor=\color{backcolour},   
    commentstyle=\color{codegreen},
    keywordstyle=\color{magenta},
    numberstyle=\tiny\color{codegray},
    stringstyle=\color{codepurple},
    basicstyle=\ttfamily\footnotesize,
    breakatwhitespace=false,         
    breaklines=true,                 
    captionpos=b,                    
    keepspaces=true,                 
    numbers=left,                    
    numbersep=5pt,                  
    showspaces=false,                
    showstringspaces=false,
    showtabs=false,                  
    tabsize=2
}

\lstset{style=mystyle}

\title{Facultad de Ciencias, UNAM \\ Criptografía y Seguridad \\ Tarea 3}
\author{Altamirano Vázquez Jesús Fernando \\
        Rubí Rojas Tania Michelle}
\date{\today}

\begin{document}
\maketitle

\begin{enumerate}
    % Ejercicio 1.
    \item Sea $\mathbb{E} : y^2 + 20x = x^3 + 21 \pmod{35}$ y sea $Q = (15, -4) 
    \in \mathbb{E}$.
    \begin{enumerate}
        % Ejercicio 1.a
        \item Factoriza $35$ tratando de calcular $3Q$.
        % Ejercicio 1.b
        \item Factoriza $35$ tratando de calcular $4Q$ duplicándolo.
        % Ejercicio 1.c
        \item Calcula $3Q$ y $4Q$ sobre $\mathbb{E} \pmod{5}$ y sobre 
        $\mathbb{E} \pmod{7}$. Explica por qué el factor $5$ se obtiene 
        calculando $3Q$ y por qué el factor $7$ se obtiene calculando $4Q$. 
    \end{enumerate}

    % Ejercicio 2.
    \item Sea $\mathbb{E}$ la curva elíptica $y^2 = x^3 + x + 28$ definida sobre 
    $\mathbb{Z}_{71}$. 
    \begin{enumerate}
        % Ejercicio 2.a
        \item Calcula y muestra el número de puntos de $\mathbb{E}$.
        % Ejercicio 2.b
        \item Muestra que $\mathbb{E}$ no es un grupo cíclico.
        % Ejercicio 2.c
        \item ¿Cuál es el máximo órden de un elemento en $\mathbb{E}$? Encuentra 
        un elemento que tenga este órden. 
    \end{enumerate}

    % Ejercicio 3.
    \item Sea $\mathbb{E} : y^2 - 2 = x^3 + 333x$ sobre $\mathbb{F}_{347}$ y 
    sea $P = (110, 136)$. 
    \begin{enumerate}
        % Ejercicio 3.a
        \item ¿Es $Q = (81, -176)$ un punto de $\mathbb{E}$?
        % Ejercicio 3.b
        \item Si sabemos que $|\mathbb{E}| = 358$. ¿Podemos decir que 
        $\mathbb{E}$ es criptográficamente útil? ¿Cuál es el órden de $P$? 
        ¿Entre qué valores se puede escoger la clave privada?
        % Ejercicio 3.c
        \item Si tu clave privada es $k = 101$ y algún conocido te ha enviado 
        el mensaje cifrado
        \begin{equation*}
            (M_1 = (232, 278), M_2 = (135, 214))
        \end{equation*}

        ¿Cuál era el mensaje original?
    \end{enumerate}

    % Ejercicio 4.
    \item Sea $\mathbb{E} : F(x, y) = y^2 - x^3 - 2x - 7$ sobre 
    $\mathbb{Z}_{31}$ con $\# \mathbb{E} = 39$ y $P = (2, 9)$ es un punto de 
    órden $39$ sobre $\mathbb{E}$, el \textsc{ECIES} simplificado definido sobre 
    $\mathbb{E}$ tiene $\mathbb{Z}^{*}_{31}$ como espacio de texto plano, 
    supongamos que la clave privada es $m = 8$. 
    \begin{enumerate}
        % Ejercicio 4.a
        \item Calcula $Q = mP$.
        % Ejercicio 4.b
        \item Descifra la siguiente cadena de texto cifrado 
        \begin{equation*}
            ((18, 1), 21), ((3, 1), 18), ((17, 0), 19), ((28, 0), 8)
        \end{equation*}
        % Ejercicio 4.c
        \item Supongamos que cada texto plano representa un carácter alfabético,
        convierte el texto plano en una palabra en Inglés. Usa la asociación 
        $(A \rightarrow 1, ..., Z \rightarrow 26)$.
    \end{enumerate}
\end{enumerate}

\end{document}